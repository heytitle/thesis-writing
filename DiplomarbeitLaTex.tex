\documentclass[bibtotoc,liststotoc,BCOR5mm,DIV12]{scrbook}

% use this declaration to set specific page margins
%\usepackage[a4paper , lmargin = {2.7cm} , rmargin = {2.9cm} , tmargin = {2.7cm} , bmargin = {4.6cm} ]{geometry}
\usepackage[a4paper]{geometry}
\usepackage{amsfonts}
\usepackage{amsmath}

\usepackage[ngerman, english]{babel}
%\usepackage{bibgerm}       		% german references
\usepackage{natbib}

\usepackage{amsthm}
\theoremstyle{definition}
\newtheorem{definition}{Definition}[section]

\usepackage[latin1]{inputenc} % german characters
\usepackage{graphicx} 				% it's recommended to use PDF images but you can use JPG or PNG as well
\usepackage{url}           		% format URLs
\usepackage{hyperref} 				% create hyperlinks
\usepackage{listings, color}	% for source code
\usepackage{subfig}						% two figures next to each other (example: figure 3a), figure 3b)
\usepackage{scrpage2}					% header and footer line
%\usepackage{ulem}
\usepackage[normalem]{ulem} %to strike the words

\usepackage{bbm}
\usepackage[ruled]{algorithm2e}
\usepackage{afterpage}
\usepackage{nomencl}
\usepackage{breakcites}

\usepackage[ddmmyyyy]{datetime}

%\usepackage{datetime}




\makenomenclature


\usepackage{mfirstuc}


% header and footer line - no header & footer line on pages where a new chapter starts
\pagestyle{scrheadings}
\ohead{Designing Recurrent Neural Networks for Explainability}
\ihead{Pattarawat Chormai}
\ofoot[]{\thepage}
\usepackage{tikz}

%TODO : Footer : What is Fachgebiet? and 
%\ifoot{Master Thesis, TU Berlin, Fachgebiet ML, 2018}

% set path where images are stored
\graphicspath{{./img/}}


\addto{\captionsenglish}{%
  \renewcommand{\bibname}{References}
}

%TODO: Define commands
\newcommand{\addfigure}[1] {Figure #1}
\newcommand{\patmatrix}[1]{\boldsymbol{\uppercase{#1}}}
\newcommand{\patvector}[1]{\boldsymbol{\lowercase{#1}}}
\newcommand{\x}{\boldsymbol{x}}
\newcommand{\btheta}{\boldsymbol{\theta}}

\newcommand{\patarg}[2]{ \underset{#2}{\text{arg#1}}\   }

%
\newcommand*\patcircle[1]{\tikz[baseline=(char.base)]{
            \node[shape=circle,draw,inner sep=2pt, black, fill=white,minimum size=1em] (char) {\textbf{\footnotesize{#1}}};}}
            
\newcommand\encircle[1]{%
  \tikz[baseline=(X.base)] 
    \node (X) [draw, shape=circle, inner sep=0] {\strut #1};}
\newcommand*\circled[1]{(#1)}
            
\newcommand{\rnncellseq}[2]{\capitalisewords{#1}-#2}
\newcommand{\rnncell}[1]{\text{\capitalisewords{#1}}}
\newcommand{\todo}[1]{\textcolor{red}{TODO : #1}}
\newcommand{\xa}[0]{\patvector{x}^{(\alpha)}}
\newcommand{\ya}[0]{y^{(\alpha)}}
\newcommand{\yha}[0]{\hat{y}^{(\alpha)}}
\newcommand{\lrpp}[0]{\text{LRP-}\alpha_{1.5}\beta_{0.5}}
\newcommand{\heatmapscaleexplain}{Blue indicates negative relevance, while red indicates positive relevance.}
\newcommand{\quantitativeplotexplain}{The values are averaged from test sets of $7$-fold cross-validation and the vertical lines depict the 95\% confidence interval. Accuracy of the models can be found at Appendix  \ref{annex:model_acc}.}

\newcommand{\patcaption}[2]{\caption[#1]{#1 #2}}
\newcommand{\setreal}[0]{\mathbb{R}}
\newcommand{\patpartial}[2]{\frac{ \partial #1}{ \partial #2 }}
\renewcommand{\dateseparator}{.}


\setcitestyle{square}
 
\input{./misc/hyphenation} 					% use this file to set explicit hyphenations (doesn't seem to work correctly)

\begin{document}
% ---------------------------------------------------------------
\frontmatter
    \thispagestyle{empty}
\begin{center}

{\LARGE \textbf{Technische Universit\"at Berlin}}

\vspace{0.5cm}

{\large Faculty IV : Electrical Engineering and Computer Science
\\[1mm]}
{\large Institute of Software Engineering and Theoretical Computer Science\\[5mm]}

Machine Learning Group\\
Marchstr. 23\\
10587 Berlin\\

\vspace*{1cm}

\includegraphics[width=4cm]{tu_logo.jpg}

\vspace*{1.0cm}

{\LARGE{Master Thesis} }

\vspace{1.0cm}
{\LARGE \textbf{Designing Recurrent Neural Networks}}\\
\vspace*{0.3cm}
{\LARGE \textbf{ for Explainability}}\\
\vspace*{1.0cm}
{\LARGE Pattarawat Chormai}
\\
\vspace*{0.5cm}
Matriculation Number: 387441\\
\today  \\ % 	date of submission

\vspace*{1.0cm}

\textbf{Supervised by}\\
Prof. Dr. Klaus-Robert M\"{u}ller \\
Dr. Gr\'{e}goire Montavon
%\vspace*{0.5cm}
%\textbf{Advisor}\\


\vspace{3cm}


\end{center}


   	\thispagestyle{empty}
    \cleardoublepage
    
    \thispagestyle{empty}
\vspace*{3cm}
\begin{center}
    \textbf{Acknowledgement}
\end{center}

\noindent
First of all, I would like to thank Prof. Dr. Klaus-Robert M\"{u}ller and Dr. Gr\'{e}goire Montavon for this research opportunity, invaluable guidance throughout conducting the thesis, and facilitating me at the Machine Learning group, TU Berlin with an inspiring research atmosphere.
\\

\noindent
Secondly, I would also like to thank Prof. Dr. Klaus Obermayer and his staffs at the Neural Information Processing group, TU Berlin for organizing Machine Intelligence I \& II and Neural Information Project courses. These courses provided me necessary knowledge to conduct this thesis.
\\

\noindent
Importantly, I would like to give special thanks to my family for always supporting me in every aspect as well as encouraging me to develop and pursue my own interests.
\\

\noindent
I would like to say thank you to my all friends for all discussions we had, especially the EIT Data Science fellows including Akash Singh, Zitong Lian, Pham Duy and Shashank Srivastava who gave me detailed feedback of the first draft of this thesis. I would also like to thank the EIT Master School, TU/e, and TUB staffs who have been organizing this wonderful master study program.
\\

\noindent
I would also like to acknowledge William Vanmoerkerke for lending me a powerful laptop throughout my study. None of my work, including assignments and this thesis, would have been achieved without this laptop.  Lastly, I would like to credit Amazon AWS for offering the educational credits, Github for the student developer pack and Sketch for the student discount. These resources were used in this thesis.
    \thispagestyle{empty}
    \cleardoublepage
    
    \newpage

\thispagestyle{empty}

\begin{large}

\vspace*{6cm}

\noindent
Hereby I declare that I wrote this thesis myself with the help of no more than the mentioned literature and auxiliary means.
\vspace{2cm}

\noindent
%TODO : Update self-assertion dage
Berlin, 15.03.2018

\vspace{3cm}

\hspace*{7cm}%
\dotfill\\
\hspace*{8.5cm}%
\textit{Pattarawat Chormai}

\end{large}
 
    \thispagestyle{empty}
    \cleardoublepage
    
    
    \thispagestyle{empty}
\vspace*{1.0cm}

\begin{center}
    \textbf{Abstract}
\end{center}

\vspace*{0.5cm}

\noindent
%TODO Writing abstract 
Standard (non-LSTM) recurrent neural networks have been challenging to train, but special optimization techniques such as heavy momentum makes this possible. However, the potentially strong entangling of features that results from this difficult optimization problem can cause deep Taylor or LRP-type to perform rather poorly due to their lack of global scope. LSTM networks are an alternative, but their gating function make them hard to explain by deep Taylor LRP in a fully principled manner. Ideally, the RNN should be expressible as a deep ReLU network, but also be reasonably disentangled to let deep Taylor LRP perform reasonably. The goal of this thesis will be to enrich the structure of the RNN with more layers to better isolate the recurrent mechanism from the representational part of the model. 
    \thispagestyle{empty}
    \cleardoublepage
%    
%    \include{./misc/abstract_de}
%    \thispagestyle{empty}
%    
    
    
    
    \tableofcontents
    \thispagestyle{empty}
    
    \thispagestyle{empty}
\addcontentsline{toc}{chapter}{Notation}

%\vspace*{0.2cm}

%\begin{center}
%    \textbf{Mathematical Notation}
%\end{center}

%\vspace*{0.2cm}
\renewcommand{\nomname}{Notation}
\setlength{\nomlabelwidth}{3cm}

\printnomenclature
\nomenclature{$a_j$}{Activation of a neuron $j$}
\nomenclature{$a_j^{(l)}$}{Activation of a neuron $j$ in a layer $l$}
\nomenclature{$b_j$}{Bias of a neuron $j$}
%\nomenclature{$x_i$}{Feature $i$ of an input sample $x$}
\nomenclature{$w_{jk}$}{Weight between a neuron $j$ and a neuron $k$} 
\nomenclature{$R_j$}{Relevance score of a neuron $j$} 
\nomenclature{$R_{j \leftarrow k }$}{Relevance score propagated from a neuron $k$ to a neuron $j$} 
\nomenclature{$\sigma$}{An activation function}
\nomenclature{$( a_j )_j, \boldsymbol{a}^{(j)}$}{A vector of activations of neurons in a layer $j$}
\nomenclature{$\patvector{x}, \patvector{x}^{(\alpha)}$}{A vector of an input sample} 
\nomenclature{$\patvector{\theta}$}{Parameters of a neural network} 
\nomenclature{$\patvector{\x_t}$}{A vector of input corresponding to a time step $t$} 
\nomenclature{$y, \ya$}{A true label of an input sample}

%\nomenclature{$R(\patvector{x})$}{Relevance heatmap of $\patvector{x}$} 
    
    \listoffigures
    \thispagestyle{empty}
    
    \listoftables
    \thispagestyle{empty}
    
    
    
% --------------------------------------------------------------

\mainmatter % comment single chapters for faster compilation

	\chapter{Introduction}
\label{cha:chapter1}
In recent years, machine learning  has been increasingly involved in almost every aspect of our life, for example recommendation systems on e-commerce sites, cancer diagnosis, or self-driving car. This developments can not be achieved without intelligent algorithms behind. Due to great amount of data and more efficient computational resources, a certain type of machine learning, called neural networks, are directly benefitted and are able to achieve much better performance that traditional machine learning techniques. As a result, intelligent systems we use nowadays rely on them somehow.


In short, NN, a learning paradigm inspired by human brain, are algorithms developed to efficient learn pattern from data. They have units, called neurons, arranged in layers working together to transform input to a desired output. When networks contains many layers, they are referred as deep learning. Connections between neurons define this transformation and will be learned from data. Because the transformation is typically in high dimensional space and built specifically to a certain problem, it is not obvious to us how trained NN utilize an input and make a prediction.  This lack of understanding  raises concerns and questions to the machine learning community and consumers. One major concern is about trust, in particular, how we can ensure that NN will work as we expect. Secondly, lacking of this understanding results in great amount of trials and errors when it comes to adjust configurations of NN to achieve expected performance.

Several methods have been proposed by researchers in order to better understand or explain how NN transform input to output. In particular, \cite{SimonyanDeepConvolutionalNetworks2013}  proposed a pioneer work in understand the predictions of NN through a method called, \textit{sensitivity analysis}(SA), as well as features that each layer in those networks learn. \cite{SpringenbergStrivingSimplicityAll2014e} suggested a modified version of SA, called \textit{guided backprop}(GB), for ReLU-type neural networks. The result demonstrated that GB produces more meaniful explanations than SA. \cite{SmilkovSmoothGradremovingnoise2017} also suggested an approach to improve quality of SA explanations. \cite{BachPixelWiseExplanationsNonLinear2015} proposed an alternative approach, called \textit{Layer-Wise Relevance Propagation}(LRP). The metho utilizes architecture of the neural network itself to create explanations, instead of relying on derivatives as in SA and GB. For ReLU-type networks,\cite{MontavonExplainingnonlinearclassification2017} showed that LRP can be equivalent to \textit{deep Taylor decomposition}(DTD). \cite{SundararajanAxiomaticAttributionDeep2017} proposed \textit{integrated gradients} combining gradient and decomposition technique. \cite{RibeiroWhyShouldTrust2016} developed \textit{Local Interpretable Model-Agnostic Explanations}(LIME) that can explain predictions from wider set of models. \cite{OlahBuildingBlocksInterpretability2018} suggested ideas for visualizing explanations from multiple domains. 

These work have primarily focused on standard NN, or feed-forward architectures, however there is still only a limited number of work in this direction applied to recurrent neural networks(RNN) which is considerably important in domain of processing sequential data, such as machine translation(MT) and natural language processing(NLP). In fact, the closest work in this area is from \cite{ArrasExplainingRecurrentNeural2017} where they applied LRP to  LSTM\cite{HochreiterLongshorttermmemory1997} trained to perform a sentiment analysis task. Therefore, a study of these explanation techniques on RNN need to be explored. This understanding study will enable us to gain insigh how each RNN internally works and hopefully it will lead us to development more explainable architectures.



%In fact, state-of-the-art of these applications today mostly rely on RNN. However, the question of how neural networks, including RNN, derive their predictions is still unclear to us. This causes resistance in the adoptation and development of the technology itself. 

%
%
%However, there is still only few work in the direction of explaining predictions from RNN. 
%
%
% Although these work has demonstrated promising results on neural networks, more specifically feed-forward architectures, there is still only a few applications of these methods to recurrent architecture. These recurrent neural networks are considerably important and have been powering almost machine learning systems processing sequential data, such as machine translation and natural language understanding. Therefore, applications of these explanation techniques to recurrent neural networks need to be explored. This results will also enable us to propose adjustments to recurrent architectures such that they are more explainable.


\section{Objective \& Scope}
This thesis aims to explain RNN predictions. More precisely, the goal of this thesis is to study how structure of RNN affect the quality of explanations produced by various explanation techniques. In particular, we are interested in applying explanation techniques, such as sensitivity analysis(SA), guided backprop(GB), Layer-Wise Relevance Propagation(LRP) and deep Taylor decomposition(DTD) to RNN. 

Our study is based on artificial classification problems that areare specifically constructed such that  knowledge of ground truth explanations is available to us. As a result, we can perform qualitative and quantitative measurements accordingly.


We have a hypothesis that RNN with more layers are more expressible than fewer layers. We extensively conduct experiments on various configurations verify our proposition. We also propose an adjustment to LSTM such that it can be explained by the techniques mentioned above.

Lastly, as the primary goal is to study explainability of RNN, it is worth mentioning that we do not seek to train models to achieve the state-of-the-art performance. We rather train them to reach a certain level of performance. We assume that models operating in this level are good enough and produce comparable explanations.


%assumption relu? ... 
%
%not care performance much ..
%
%our hypothesis ... 
%
%artificial problems ... 


\section{Dataset}
Our study is based on MNIST\cite{LeCunMNISThandwrittendigit2010} and FashionMNIST\cite{XiaoFashionMNISTNovelImage2017}. Following is a brief introduction of them.

\subsection{MNIST}

\begin{figure}[!hbt]
	\centering
	\includegraphics[width=0.6\textwidth]{mnist}
	\caption{MNIST dataset.}
	\label{fig:mnist_samples}
\end{figure}

MNIST is one of the most popular dataset that machine learning partitioners use to benchmark machine learning algorithms. The dataset consists of 60,000 training and 10,000 testing samples. Each sample is a grayscale 28x28 image. As shown in \addfigure{\ref{fig:mnist_samples}}, MNIST has 10 categories corresponding to a digit between 0 and 9.


State-of-the-art algorithms can classify MNIST with accuracy higher than 99\%, while classical ones, such as SVC or RandomForest, are able to achieve around 97\%\cite{XiaoFashionMNISTNovelImage2017}.


\subsection{FashionMNIST}

FashionMNIST is a dataset whose authors aim to it to be a replacement of MNIST, especially in  benchmarking machine learning algorithms.  According to \cite{XiaoFashionMNISTNovelImage2017},  FashionMNIST is more representative to modern computer vision tasks. It contains images of fashion products from 10 categories and compatible  to MNIST in every aspects, such as the size of training and testing set, image dimension and data format, hence one can easily  apply existing algorithms that work with MNIST to Fashion-MNIST without any change. \addfigure{\ref{fig:fashion_mnist_samples}} shows FashionMNIST samples.

\begin{figure}[!hbt]
\centering
\includegraphics[width=0.65\textwidth]{sketch/fmnist_samples}
\caption{FashionMNIST dataset.}
\label{fig:fashion_mnist_samples}
\end{figure}

\cite{XiaoFashionMNISTNovelImage2017} also reports benchmarking results of classical machine learning algorithms on Fashion-MNIST. On average, they achieve accuracy between 85\% to 89\%. According to Fashion-MNIST's page\footnote{https://github.com/zalandoresearch/fashion-mnist}, state-of-the-art result has 97\% accuracy using Wide Residual Network(WRN)\cite{ZagoruykoWideResidualNetworks2016} and standard data preprocessing and augmentation.

%\section{Terminology}
%\todo{terminology}

\section{Outline}
The thesis is organized as follows :
\begin{itemize}
%	\item \textbf{Chapter \ref{cha:chapter2}} provides a brief literature survey and related work in the direction towards explaining neural networks.
	\item \textbf{Chapter \ref{cha:chapter3}} summaries relevant topics in  neural networks and explanation techniques that are studied in the thesis.
	\item \textbf{Chapter \ref{cha:chapter4}} is devoted to experimental results.
	\item \textbf{Chapter \ref{cha:chapter5}} concludes the results and discusses challenges as well as future work.
\end{itemize}

%
%This chapter should have about 4-8 pages and at least one image, describing your topic and your concept. Usually the introduction chapter is separated into subsections like 'motivation', 'objective', 'scope' and 'outline'.
%
%\section{Motivation\label{sec:moti}}
%
%Start describing the situation as it is today or as it has been during the last years. 'Over the last few years there has been a tendency... In recent years...'. The introduction should make people aware of the problem that you are trying to solve with your concept, respectively implementation. Don't start with 'In my thesis I will implement X'.
%
%\section{Objective\label{sec:objective}}
%
%What kind of problem do you adress? Which issues do you try to solve? What solution do you propose? What is your goal?
%'This thesis describes an approach to combining X and Y... The aim of this work is to...'
%
%\section{Scope\label{sec:scope}}
%
%Here you should describe what you will do and also what you will not do. Explain a little more specific than in the objective section. 'I will implement X on the platforms Y and Z based on technology A and B.'
%
%Conclude this subsection with an image describing 'the big picture'. How does your solution fit into a larger environment? You may also add another image with the overall structure of your component.
%
%'Figure \ref{fig:intro} shows Component X as part of ...' 
%\\
%\begin{figure}[htb]
%  \centering
%  \includegraphics[width=9cm]{intro_example.pdf}\\
%  \caption{Component X}\label{fig:intro}
%\end{figure}
%
%\section{Outline\label{sec:outline}}
%
%The 'structure' or 'outline' section gives a brief introduction into the main chapters of your work. Write 2-5 lines about each chapter. Usually diploma thesis are separated into 6-8 main chapters. 
%\\
%\\
%\noindent This example thesis is separated into 7 chapters.
%\\
%\\
%\textbf{Chapter \ref{cha:chapter2}}  Neural Network and Explainability foudation
%%is usually termed 'Related Work', 'State of the Art' or 'Fundamentals'. Here you will describe relevant technologies and standards related to your topic. What did other scientists propose regarding your topic? This chapter makes about 20-30 percent of the complete thesis.
%\\
%\\
%\textbf{Chapter \ref{cha:chapter3}} Architecture
%%analyzes the requirements for your component. This chapter will have 5-10 pages.
%\\
%\\
%\textbf{Chapter \ref{cha:chapter4}} Experiments
%%Ais usually termed 'Concept', 'Design' or 'Model'. Here you describe your approach, give a high-level description to the architectural structure and to the single components that your solution consists of. Use structured images and UML diagrams for explanation. This chapter will have a volume of 20-30 percent of your thesis.
%\\
%\\
%\textbf{Chapter \ref{cha:chapter5}} Conclusion and future work
%%describes the implementation part of your work. Don't explain every code detail but emphasize important aspects of your implementation. This chapter will have a volume of 15-20 percent of your thesis.
%%\\
%\\
%%\textbf{Chapter \ref{cha:chapter6}} is usually termed 'Evaluation' or 'Validation'. How did you test it? In which environment? How does it scale? Measurements, tests, screenshots. This chapter will have a volume of 10-15 percent of your thesis.
%%\\
%%\\
%%\textbf{Chapter \ref{cha:chapter7}} summarizes the thesis, describes the problems that occurred and gives an outlook about future work. Should have about 4-6 pages.
	
	
	\chapter{Background}\label{cha:chapter3}

\section{Neural Networks}
Neural networks(NN) are a type of machine learning algorithms that are inspired by how human brain works.  In particular, NN have units called neurons connecting together similar to the way neurons in our brain do. These connections allow NN to build hierarchical representations that are necessary to perform an objective task. \addfigure{\ref{fig:nn_simple}} illustrates a reaction task that neurons in our brain perform together to achieve.

 \begin{figure}[ht!]
    \begin{center}

\includegraphics[width=\textwidth]{nn_simple}
\caption{An illustration of how neurons in human brain cooperate together to sense the pain and react accordingly.}
\small{Source : \cite{LeonMakingSimpleNeural2017}}
\label{fig:nn_simple}

\end{center}
\end{figure}

 \begin{figure}[ht!]
    \begin{center}

\includegraphics[width=0.8\textwidth]{sketch/typical_nn_structure}
\caption[]{A general structure of neural networks and details of a neuron's connectivity and activity.}
\label{fig:nn_typical_structure}

\end{center}
\end{figure}


% \begin{figure}[ht!]
%	\begin{center}
%
%		\includegraphics[width=0.5\textwidth]{sketch/a_neuron}
%		\caption{Connectivity and activity of a neuron}
%		\label{fig:a_neuron}
%	\end{center}
%\end{figure}


\addfigure{\ref{fig:nn_typical_structure}} illustrates a general structure of NN. The network has input layer, output layer and hidden layers, which can be analogously viewed to sensory, motor and inter neurons in \addfigure{\ref{fig:nn_simple}}. The figure also shows connections of a neuron to other neurons in previous and following layer. Given an objective task, the goal is to construct connections between these neurons such that the network can transform an input sample into a desired output.  This connections are determined by trainable variables, called weights and biases. They are denoted by $w_{i\rightarrow j}, w_{j\rightarrow k}$ and $b_j$ respectively in the figure. These variables will be learned  during \textit{training} process.  In this example, the objective task is to classify what is the given digit.


Consider a given set of $p$ training samples $\mathcal{D} = \{ \xa, \ya) \}_{\alpha=1}^{p}$,  there are 3 primary components to build a NN, namely  
\begin{enumerate}
	\item \textbf{Network architecture} defines the configuration of NN. Some of important settings are number of layers and number of neuron in each layer and type of activation function. These settings determine neurons' activity and how they communicate to each other through connections governed by trainable weights and biases. Typically, the weights and biases are denoted as $\patvector{\theta}$. Mathematically, NN can be viewed as a function $f$ with parameters $\patvector{\theta}$ that nonlinearly transforms an input $\xa\in \mathbb{R}^d $ to some output $f(\xa)$.
	\item \textbf{Loss function $L$}  is a measurement corresponding to the objective task. It quantifies how far NN output $f(\xa)$ is the true output $\ya$. Loss averaged over training samples is a major contributor to\textit{Cost} function $J$, a function that describes the objective of learning. Regularization is another term in $J$. \todo{wrtiing Regularizationl}.
	\item \textbf{Learning algorithm} is responsible for optimizing trainable variables in the network such that the the cost function is minimized. Practically, we learn this variables through optimizing the cost of training samples \textit{Empirical Error}. This is a proxy to optimize the cost of the data distribution(\textit{Generalization Error}). 
\end{enumerate}

Hence, the goal of this empirical training process can be summarized as follows : 

\begin{align} \label{eq:nn_opt}
	\patvector{\hat{\theta}} = \patarg{min}{\theta} \underbrace{\frac{1}{p}  \sum_{\alpha=1}^p L( f(\xa), \ya) }_{J}
\end{align}

\subsection{Loss functions}
Choosing loss function is depend on the objective that the network is being trained to solve. For classification problems, such as digit classification, whose goal is to categorize $\x$ into a category $C$ from $K$ categories, $f : \x \in \mathbb{R}^d  \mapsto C \in \{ C_k \}_{k=1}^K$, \textit{cross entropy} is the loss function for this purpose.
$$
L_{\text{CE}} = - \sum_{i} y_k \log \hat{y}_k,
$$
where $y_i \in [0, 1]$ and $\hat{y}_i \in [0, 1]$ are true and predicted probability that $\x$ belongs to $C_k$ respectively. Denote $\patvector{z} = f(\x) \in \mathbb{R}^{K}$. $\hat{y}_k$ is computed via \textit{softmax} function :
$$
\hat{y}_k = \frac{e^{z_k}}{ \sum_{k=1}^K{e^{z_k}} }
$$ 

For regression problems, $ f : \x \in \mathbb{R}^d  \mapsto \mathbb{R}$, such as price forecast,\textit{Mean Squared Error}(MSE) is the loss function.
$$
L_{\text{MSE}} = (f(\x) - y)^2
$$

This is a brief introduction to loss functions widely used in machine learning. More loss functions do exist and are beyond scope of the thesis to cover.

\subsection{Learning Algorithm : Gradient Descent and Backpropagation}
Due to substantial number of trainable variables in a neural network, it is crucial to solve the optimization (\ref{eq:nn_opt}) efficiently. The answer to  this high dimensional problem is to use a repeated procedure, called \textit{Gradient Descent}.  \addfigure{\ref{fig:gradent_descent_toy}} provides an intuition of the method. Consider a  function $J(\theta)$ on the figure as a toy example of a cost function of a NN with parameter $\theta$. The figure shows that if we gradually adjust $\theta$ in the opposite direction of gradient, $	-\frac{d L(\theta)}{d \theta}$,  with a proper step size $\lambda$ (\textit{learning rate}), we will eventually reach one of the minimals. This adjustment  is formally summarized in (\ref{eq:gradient_update}).


% In this case, $\hat{\theta}$ can trivially computed by solving
%
%\begin{align}
%	\frac{d L(\theta)}{d \theta}  \stackrel{!}{=} 0
%	\label{eq:simple_solve_for_thetha}
%\end{align}

\begin{figure}[!hbt]
    \begin{center}
		\includegraphics[width=0.6\textwidth]{sketch/gradient_intuition}
		\caption[]{An illustration of Gradient Descent}
		\label{fig:gradent_descent_toy}
	\end{center}
\end{figure}

\begin{align}
 \theta_i \leftarrow \theta_i - \lambda  \frac{\partial J }{\partial \theta_i}, \forall \theta_i \in \btheta
\label{eq:gradient_update}
\end{align}


Let's consider again the NN shown in \addfigure{\ref{fig:nn_typical_structure}}. Assume that the network uses an activation function $\sigma$ and has $\btheta = \{ \forall i,j,k,l : w^{(1)}_{i \rightarrow j}, w^{(2)}_{j \rightarrow k}, w^{(3)}_{k \rightarrow l}  \}$ with biases omitted. For a sample and its true target variable $(\xa, \ya)$, $f(\xa )$ can be  calculated as follows

\begin{align*}
		h_j^{(1)} &= \sum_i w_{i \rightarrow j}^{(1)} x_i^{(\alpha)} & a_j^{(1)} &= \sigma (h_j^{(1)})	\\
		h_k^{(2)} &= \sum_j w_{j \rightarrow k}^{(2)} a_j^{(1)}  & a_k^{(2)} &= \sigma (h_k^{(2)})	 \\
		h_l^{(3)} &= \sum_k w_{k \rightarrow l}^{(3)} a_k^{(2)} & a_l^{(3)} &= \sigma (h_l^{(3)})	 \\
		f(\x) &= [ a_1^{(3)}, \dots, a_L^{(3)}  ]^T  & J &= \frac{1}{p} \sum_{\alpha = 1}^{p} L(f(\xa), \ya)	
\end{align*}

The gradients can be then computed by recursively applying chain rule from the last to the first layer, yields \textit{backpropagation} algorithm.
\begin{align}
	\frac{\partial l(f(\xa), \ya)  }{\partial w_{k \rightarrow l}^{(3)} } &= 	\frac{\partial l(f(\xa), \ya) }{\partial a_{l}^{(3)} }  \frac{\partial a_{l}^{(3)} }{\partial w_{k \rightarrow l}^{(3)} }  	\\
		&= 	\underbrace{\frac{\partial l(f(\xa), \ya) }{\partial a_{l}^{(3)} } \sigma'(h_l^{(3)})}_{ \delta_l^{(3)}} a_{k}^{(2)} 	\\
	\frac{\partial l(f(\xa), \ya)  }{\partial w_{j \rightarrow k}^{(2)} } 
		&=  \sum_{l' = 1}^{L} 	\frac{\partial l(f(\xa), \ya) }{\partial a_{l'}^{(3)} } \frac{\partial a_{l'}^{(3)}}{\partial w_{j \rightarrow k}^{(2)}} \\
		&= \sum_{l' = 1}^{L} 	\frac{\partial l(f(\xa), \ya) }{\partial a_{l'}^{(3)} } \sigma'(h_{l'}^{(3)})  \frac{\partial h_{l'}^{(3)} }{\partial w_{j \rightarrow k}^{(2)}} \\
		&= \sum_{l' = 1}^{L} 	\delta_{l'}^{(3)}  w_{k \rightarrow l'}^{(3)} \frac{\partial a_{k}^{(2)} }{\partial w_{j \rightarrow k}^{(2)}} \\
		&= a_{j}^{(1)}  \underbrace{\sigma'(h_{k}^{(2)}) \sum_{l' = 1}^{L} 	\delta_{l'}^{(3)} w_{k \rightarrow l'}^{(3)}}_{\delta_{k}^{(2)}}  \\
	\frac{\partial l(f(\xa), \ya)  }{\partial w_{i \rightarrow j}^{(1)} } &=  x_i  \sigma'(h_{j}^{(1)}) \sum_{k' = 1}^{K} 	\delta_{k'}^{(2)} w_{j \rightarrow k'}^{(2)} 
\end{align}
As shown in the derivations above, backpropagation allows us to efficiently compute the gradients by reusing already calculated gradients from the later layer, for example $\delta_l^{(3)}, 	\delta_{k}^{(2)}$. Moreover, these reused quantities can be also interpreted as error propagated to responsible neurons.

In practice, because the training set usually contains several thousand samples, the gradient update in  (\ref{eq:gradient_update}) would require significant amount of computation to update just one step, not to mention that it could also result in small gradient update step leading to slow progress towards a desire objective performance. Therefore, the training data $D$ is equally divided into batches  $\widetilde{D}_i$  and perform the gradient update for every $\widetilde{D}_i$. For example, the size of $\widetilde{D}_i$ is usually chosen between 32 and 512 samples. This is referred  as \textit{mini-batch gradient descent}.

Lastly, because noise in training data and potentially highly non-smooth of the cost function, learning rate $\lambda$ has great influential on the training process. More precisely, it should not be too small or too large. This requires some effort and experience in order to get the value right. Some work have proposed alternative update rules aiming to make the training process more stable. For example,  \textit{Adaptive Moment Estimation}(Adam)\cite{KingmaAdamMethodStochastic2014}  uses an adaptive learning rate  and incorporates accumulated direction and speed of the previous gradients(\textit{momentum}) into the adjustment, hence more consistent gradient and fast convergence. Other similar proposals are RMSProp\cite{TielemanLectureRmsPropDivide2012} and Adadelta\cite{ZeilerADADELTAAdaptiveLearning2012}.


\subsection{Convolutional Neural Networks} \label{sec:conv}
Convolutional neural networks(CNN) refer to neural networks that employ convolutional operators to process input instead of fully-connected layers(weighted sum). Typically, a convolutional operator is followed by a pooling operator. Using this convolutional and pooling operators allows the neural network to extract hierarchical features that are spatially invariant \cite{ZeilerVisualizingUnderstandingConvolutional2013}. Hence, these type of layers increases predictive capability of NN, while using fewer parameters than traditional fully-connected layers.

%\afterpage{
\begin{figure}[!hbt]
    \begin{center}
		\includegraphics[width=0.5\textwidth]{sketch/cnn_hierachical_features}
		\caption{Hierarchical features learned by a CNN.}
		\label{fig:conv_intuition}
		\small{Source : \cite{LeeConvolutionalDeepBelief2009}}
	\end{center}
\end{figure}


\addfigure{\ref{fig:conv_intuition}} illustrates hierarchical structures that neurons in each layer of a CNN learn to detect. More precisely, this example shows that neurons in the first learn to detect low level features, such as edges, and neurons in middle layer then use knowledge to detect higher level features, for example, nose, mouth or eyes.


%\afterpage{
\begin{figure}[!hbt]
    \begin{center}
		\includegraphics[width=0.8\textwidth]{lenet}
		\caption{LeNet-5 architecture for a digits recognition task.}
		\label{fig:lenet}
		\small{ Source : \cite{LeCunGradientBasedLearningApplied2001} }
	\end{center}
\end{figure}


Since \cite{LeCunGradientBasedLearningApplied2001} proposed LeNet-5, shown in \addfigure{\ref{fig:lenet}}, and successfully applied it to handwritten recognition problems, CNN have become the first choice of architectures in many domains. Particularly, in computer vision, CNN are the major component of state-of-the-art results in various contests. Such successful results are :  AlexNet\cite{KrizhevskyImageNetClassificationDeep2012} that archive  remarkable results on  ImageNet Large-Scale Visual Recognition Challenge 2012(ILSVRC 2012) followed by the achievement of VGG\cite{SimonyanVeryDeepConvolutional2014} and GoogleLenet \cite{SzegedyGoingDeeperConvolutions2014} architecture in ILSVRC 2014 and ResNet\cite{HeDeepResidualLearning2015} in ILSVRC 2015.



\subsection{Recurrent Neural Networks}
Recurrent neural networks(RNNs) are neural networks whose outputs are repeatedly incorporated back into its next computation. \addfigure{\ref{fig:rnn_unfold}} illustrates this idea of recurrent computation by unfolding RNN into computation steps. Let's consider $\x$ a sequence of $\{ x_1, \dots, x_T \}$.  At step $t$, RNN takes $r_{t-1}$ and $x_{t}$ to compute $r_{t}$ and $\hat{y}_t$. This recurrent connections can be interpreted as accumulating information from the past, hence RNN are well suit to process sequential data, with no limitation in length . Natural Language Processing(NLP) and Machine Translation(MT) are some of the fields that RNN are widely applied to.


\begin{figure}[!hbt]
\centering
\includegraphics[width=0.8\textwidth]{sketch/rnn_unfold}
\caption{Unfolded RNN Structure}
\small{Inspired by a figure in \cite{OlahUnderstandingLSTMNetworks2015} }
\label{fig:rnn_unfold} 
\end{figure}


\subsubsection{Backpropagation Through Time}
As the number of computation steps in RNN is depend on the length of samples, which can be different in principle, one needs to organize data in such a way that samples in the same batch have the same  length of computation before training a RNN. As a result, training RNNs can be viewed as training a feedforward neural network with a certain depth of layers, hence backpropagation can be readily applied. In this case, the difference between NN and RNN is the fact that variables are shared the same across layers.

Consider again RNN in \addfigure{\ref{fig:rnn_unfold}} with $\x = \{x_1, \dots, x_T \}$ and $r_0 = 0$. Assume that only $\hat{y}_T$ determines the value of the loss function. The computations are also defined as follows 
\begin{align}
	h_1 &= w_{rx} x_1 + w_{rr} r_0 & r_1 &= \sigma(h_1) \label{eq:naive_r} \\
	h_2 &= w_{rx} x_2 + w_{rr} r_1 &  r_2 &= \sigma(h_2) \\
	& \vdots & \vdots \\
	h_{T-1} &= w_{rx} x_{T-1} + w_{rr} r_{T-2} &  r_{T-1} &= \sigma(h_{T-1}) \\
	\hat{y} &= \sigma(w_{yx} x_T   + w_{yr} r_{T-1})
\end{align}

Therefore, the gradients can be computed by 
\begin{align}
	\frac{\partial l}{\partial w_{yx}} &= \sigma'(w_{yx} x_T   + w_{yr} r_{T-1}) x_T \\
	\frac{\partial l}{\partial w_{yr}} &= \sigma'(w_{yx} x_T   + w_{yr} r_{T-1}) r_{T-1} \\
	\frac{\partial l}{\partial w_{rx}} &= 	w_{yr} \sigma'(w_{yx} x_T   + w_{yr} r_{T-1}) \frac{\partial r_{T-1}}{\partial w_{rx}} \\
	&= w_{yr} \sigma'(w_{yx} x_T   + w_{yr} r_{T-1})  \Bigg[ \sigma'(h_{T-1}) \bigg( x_{T-1} + w_{rr}  \frac{\partial r_{T-2}}{\partial w_{rx}} \bigg) \Bigg]  \label{eq:gradient_wrr}  \\
	\frac{\partial l}{\partial w_{rr}} &= w_{yr} \sigma'(w_{yx} x_T   + w_{yr} r_{T-1})  \frac{\partial r_{T-1}}{\partial w_{rr}}  \\
	&= w_{yr} \sigma'(w_{yx} x_T   + w_{yr} r_{T-1})  \Bigg[ \sigma'(h_{T-1}) \bigg( \frac{\partial r_{T-2}}{\partial w_{rr}} \bigg) \Bigg] \\
\end{align}
However, as the computations unfolded, we can see that there are 2 problems that might happen to the gradients of the shared parameters $w_{rx}$ and $ w_{rr}$, namely
\begin{itemize}
	\item \textit{Exploding gradient} happens if the gradient is derived from shared weights, for example $w_{rr}$ in  (\ref{eq:gradient_wrr}), whose absolute value is greater than one. The recursive multiplication will result in a large value of the gradient leading to unreliable training. \cite{PascanuUnderstandingexplodinggradient2012} have proposed \textit{gradient clipping} to alleviate the problem.
	\item \textit{Varnishing gradient}  in contrast occurs when the values are smaller than one yielding  near zero gradients. This issue leads to slow learning, hence RNN would require enormous of time to learn, especially long term dependencies. The next section discusses techniques to mitigate this problem.
\end{itemize}


\subsubsection{Long Short-Term Memory and Gated RNNs}
Varnishing gradient is a major problem that causes RNN to learn long term memories with slow progress. This is because the computation of recurrent connections are constructed. In particular, as in (\ref{eq:naive_r}), standard RNN compute those connections with weighted sum at every step $t$ leading to recursive multiplication terms in the gradient computations.

Alternatively, \cite{HochreiterLongshorttermmemory1997} proposed \textit{Long Short-Term Memory}(LSTM) architecture that employs a gating mechanism and additive updates in the computation of the recurrent connections. This mechanism decreases number of damping factors involved in the gradients, hence enabling better learning long term memories.

As shown in \addfigure{\ref{fig:lstm_structure}}, LSTM utilizes 3 gates, namely input $i_g$, forget $f_g$ and $o_g$ output gate, to control  information flow through LSTM cell. More precisely, $i_g$ and $f_g$ decides how to accumulate information from the previous cell state $C_{t-1}$, and the cell state computed from previous output $h_{t-1}$ and current input $x_t$, into a new cell state $C_t$. On the other hand, $o_g$ determines to leakage of the information from $C_t$ to outside $h_t$. Formally, 
\begin{align}
	i_g &= \sigma( w_{ix} x_t + w_{ih} h_{t-1} )  &  	f_g &= \sigma( w_{fx} x_t + w_{fh} h_{t-1} )\\
	o_g &= \sigma( w_{ox} x_t + w_{oh} h_{t-1} ) & \widetilde{C}_t &= \tanh(w_{cx} x_t + w_{ch} h_t) \\
	C_t &= f_g \otimes C_{t-1} + i_g  \otimes  \widetilde{C}_t & h_{t} &= o_g \otimes \tanh(C_t)
\end{align}


\begin{figure}[h]
\centering
\includegraphics[width=1\textwidth]{sketch/lstm}
\caption{LSTM Structure} 

\label{fig:lstm_structure} 
\end{figure}

Since the work was published, LSTM has successfully contributed to many state-of-the-art results in machine translation and NLP\cite{MelisStateArtEvaluation2018}. \cite{GreffLSTMsearchspace2017} demonstrated that the forget and output gate are the crucial parts of LSTM.  \cite{ChoLearningPhraseRepresentations2014a} proposed \textit{Gated Recurrent Unit}(GRU) that employs only 2 gates, however \cite{Jozefowiczempiricalexplorationrecurrent2015a} conducted several benchmarking tasks and found no significant difference in performance between LSTM and GRU. 
	\section{Explainability of Neural Networks}
Neural networks have become one of major machine learning algorithms used in many applications, for example computer vision and medicine. Despite those achievements, they are still considered as a blackbox process whose results are hardly to be interpreted by human. In particular, we barely know evidences how the networks transform input to such accurate decisions.


%\begin{figure}
%  \centering
%  \begin{minipage}{\textwidth}
%  
%			\includegraphics[width=0.6\textwidth]{sketch/husky_explanation}
%
%    \caption[Compact Routing Example]%
%    {Compact Routing\footnote{something} Example}
%  \end{minipage}
%\end{figure}
%


 \begin{figure}[!hbt]
	 		\centering
			\includegraphics[width=0.6\textwidth]{sketch/husky_explanation}
			\caption{A classifier classifies ``husky" as ``wolf" because of the snow background.}
			  \small{ Source : \cite{RibeiroWhyShouldTrust2016} }
			\label{fig:husky_explanation}
\end{figure}
%\footnotetext{}


Practically, it is always important to verify whether trained neural networks properly utilize data or what information they use to make decisions. From literatures, This kind of functional understanding is typically referred to \textit{explaining} or \textit{interpreting} prediction: neural networks are explainable if their predictions can be associated back to what relevant in the input. \cite{BachAnalyzingclassifiersFisher2016, RibeiroWhyShouldTrust2016} demonstrated cases where neural networks exploit artifacts in the data to make decisions. \addfigure{\ref{fig:husky_explanation}} is such an example. This discoveries emphasize the importance of having explainable models, not to mention that nowadays neural networks have been increasinly involved in several aspects of human life, ranging from medical development to self-driving car. 

\subsection{Global and Local Explanation}\label{sec:global_local_explanation}
Formally, there are 2 aspects of explaining neural networks, namely \textit{global} and \textit{local} explanation. Consider an image  classification problem of $\mathcal{C}$ classes, global explanation aims to find an image $\patvector{x}^*$ that is the most representative to a class $c_i \in \mathcal{C}$. Activation Maximization\cite{ErhanUnderstandingRepresentationsLearned2010} is one of the method in this category.

\begin{align}
\patvector{x}^*  = \patarg{max}{\patvector{x}}  \mathbb{P}( c |\patvector{x},\theta)
\end{align}


On the other hand, local explanation focuses on finding relevant information in $\patvector{x}$ that can explain why the neural network predicts that $\x$ into a certain class $c_i \in \mathcal{C}$.  More precisely, this aspect seeks to assign each pixel $x_i \in \patvector{x}$ with a score that quantitatively explains how the pixel influences the decision of the neural network. The score is formally referred as \textit{relevance score} and denoted with $R_i(\x)$ or $R_i$ if it is clear from the context. As a result, combining $R_i(\x)$ together will result in what so called, \textit{explanation}, \textit{explanation heatmap} or \textit{relevance heatmap}.

As illustrated in \addfigure{\ref{fig:comparision_between_global_and_local_analysis}}, the difference between global and local explanation can be analogously described by formulating questions as follows
\begin{itemize}
	\item Global explanation : ``How does a usual lamp look like?"
    \item Local explanation : ``Which part of the image make it look like a lamp?" 
\end{itemize}

 \begin{figure}[!hbt]
\centering
\includegraphics[width=0.8\textwidth]{sketch/global_and_local_method}
\caption{Comparison between global and local explanation}
\small{
The analogy is borrowed from Prof. M\"{u}ller's lecture slide. \\The images were taken from \cite{NguyenSynthesizingpreferredinputs2016a, BachPixelWiseExplanationsNonLinear2015}.
}
\label{fig:comparision_between_global_and_local_analysis}
\end{figure}

In the following, we are going to discuss approaches in local explanation in details and leave content of global explanation aside due to scope of the thesis. In particular, we are going to introduces these local explanation methods, namely \textit{sensitivity analysis}, \textit{guided backprop}, \textit{simple Taylor decomposition}, \textit{Layer-wise Relevance Propagation} and \textit{deep Taylor decomposition}.


%Consider $f(\patvector{x})$ is an output from a neural network classifier that is corresponding to the class prediction, for example the value at the final layer before applying softmax function.
%\begin{definition} Conservation Property
%\begin{align*}
%	\forall \patvector{x} : f(\patvector{x}) = \sum_i R_i
%\end{align*}
%Sum of relevance score of each pixel $x_i$ should equal to the total relevance that the network outputs.
%
%\end{definition}
%\begin{definition} Positivity Property
%\end{definition}
%\begin{definition} Consistency
%\end{definition}

\subsection{Sensitivity Analysis}
Sensitivity analysis(SA)\cite{SimonyanDeepConvolutionalNetworks2013} is a local explanation technique that derives relevance score $R_i(\x)$ through the  partial derivative of $f(\patvector{x})$ respect to $x_i$. More formally, it is formulated as follows 

\begin{align}
	R_i(\x) =
	 \bigg( \frac{\partial f(\patvector{x})}{ \partial x_i } \bigg)^2
\end{align}

This formulation is associated to
\begin{align}
	\sum_i R_i(\x) = || \nabla f(\patvector{x}) ||^2
\end{align}

The derivation of $\sum_i R_i(\x)$ above implies that SA seeks to explain $R_i(\x)$ from the aspect of variation magnitudes of $f(\x)$, which might not reflect the actual influence of the pixels to the decision.

Nonetheless, this technique can be easily implemented in modern deep learning frameworks, such as TensorFlow\cite{AbadiTensorFlowLargeScaleMachine2016}, via automatic differentiation. Hence, one might consider it as a first tool towards explaining neural network decisions.

\subsection{Guided Backpropagation}
Guided backpropagation(GB) is a extended version of SA where gradients are propagated throughout the network in a controlled manner. It is specifically designed for neural networks that rely on ReLU activations. \cite{SpringenbergStrivingSimplicityAll2014e} formally proposed an alternative definition  of ReLU function as
\begin{align}
	\sigma(a_j) = a_j \mathbbm{1}[ a_j > 0 ],
\end{align}
where $\mathbbm{1}[ \cdot ]$  is an indicator function, hence a new derivative of a ReLU neuron $j$ 
\begin{align}
	\frac{\partial_* f(\patvector{x}) }{ \partial a_j } = \mathbbm{1}\bigg[a_j > 0 \bigg] \mathbbm{1}\bigg[ \frac{ \partial f(\patvector{x}) }{ \partial a_j } > 0 \bigg] \frac{ \partial f(\patvector{x}) }{ \partial a_j } 
\end{align}
From the derivation above, both $\mathbbm{1}[ \cdot ]$ control whether original gradients will be propagated back. In particular, the gradients will be propagated to the previous layer only if neuron $j$ is active and the gradient from the next layer is positive. Similar to SA, the relevance score for each pixel is computed as 

\begin{align}
	R_i(\x) = \bigg( \frac{ \partial_* f(\patvector{x}) }{ \partial x_i }  \bigg)^2
\end{align}
%
%With this result, one can see that $x_i$ is relevant to the problem if activations $a_j$ that it supplies are active and positively contribute to $f(\patvector{x})$.

\subsection{Layer-wise Relevance Propagation}
The methods mentioned so far derive $R_i(\x)$ directly from $f(\patvector{x})$ and do not rely on any knowledge related to the neural network itself, such as architecture or activation values. Alternatively, \cite{BinderLayerWiseRelevancePropagation2016} proposed Layer-wise Relevance Propagation(LRP) framework that leverages such information when distributing relevance scores to $x_i$. In particular, LRP propagates relevance scores backward from layer to layer, similar to backpropagation algorithm, but just different quantities.




 \begin{figure}[!hbt]
	\begin{center}
		\includegraphics[width=0.8\textwidth]{sketch/lrp_graph}
		\caption{An illustration of relevance propagation in LRP.}
		\small{
			Source : \url{http://heatmapping.org}
			}
		\label{fig:lrp_graph}
		%todo fig : make "i" to the first layer
	\end{center}
\end{figure}
%\footnotetext{}}
%}

Consider the neural network illustrated in \addfigure{\ref{fig:lrp_graph}}. $R_j$ and $R_k$ are relevance score of  neurons $j,k$ in successive layers.  LRP provides a general form of relevance propagation as 

\begin{align} \label{eq:general_lrp_rj}
	R_j = \sum_{k} 	\delta_{j\leftarrow k} R_{k} ,
\end{align}

where $\delta_{j\leftarrow k}$ defines a proportion that  $R_{k}$ contributes to $R_j$. Consider further that activity $a_k$ of neuron $k$ is computed by 
\begin{align}
	a_k = \sigma \bigg( \sum_{j} w_{jk} a_j + b_k \bigg),
\end{align} 
where $\sigma$ is a activation function, $w_{jk}, b_k$ are the corresponding weight and bias between neuron $j$ and $k$. For monotonic increasing $\sigma$, $\delta_{j\leftarrow k}$ can be calculated as follows 
\begin{align} \label{eq:delta_j_k}
	\delta_{j\leftarrow k} = \alpha\frac{a_j w_{jk}^+}{\sum_{j} a_jw_{jk}^+} - \beta\frac{a_j w_{jk}^-}{\sum_{j} a_jw_{jk}^-},
\end{align}
where $w_{jk}^+$, $w_{jk}^-$ are $\max(0, w_{jk})$, $\min(0, w_{jk})$, and $\alpha$,  $\beta$ are parameters with $\alpha-\beta = 1$ constraint. Combining (\ref{eq:general_lrp_rj}) and (\ref{eq:delta_j_k}) results in LRP-$\alpha\beta$ rule, 

\begin{align}
	R_j = \sum_{k} 	\bigg( \alpha\frac{a_j w_{jk}^+}{\sum_{j} a_jw_{jk}^+} - \beta\frac{a_j w_{jk}^-}{\sum_{j} a_jw_{jk}^-} \bigg )  R_{k}
\end{align}

LRP procedures are summarized in Algorithm \ref{algo:lrp}.

\begin{algorithm}[H]
$f(\patvector{x}), \{ \{a\}_{l_1}, \{a\}_{l_2}, \dots, \{a\}_{l_n}\}$ = \text{forward\_pass}($\patvector{x}, \patvector{\theta}$)\;
$R_k = f(\patvector{x})$\;
 \For{ $\text{layer} \in \text{reverse}(\{l_1, l_2, \dots, l_n\})$}{
$ \text{prev\_layer} \leftarrow \text{layer}  - 1$ \;
	\For{ $j \in $ neurons(prev\_layer), $k \in$ neurons(layer)}{
		$R_j \leftarrow \text{LRP-}\alpha\beta(R_k, \{a\}_j, \{ w \}_{j,k} )$
	}
 }
 \caption{LRP Algorithm}
 \label{algo:lrp}
\end{algorithm}


Alternatively, if we slightly rearrange  the rule to 
$$
	R_j = \sum_{k} \bigg( \frac{a_j w_{jk}^+}{\sum_{j} a_jw_{jk}^+} \hat{R}_{k} + \frac{a_j w_{jk}^-}{\sum_{j} a_jw_{jk}^-} \check{R}_{k} \bigg),
$$ 
where $\hat{R}_{k}  = \alpha R_{k}$ and  $\check{R}_{k} = -\beta R_{k} $. We can then intuitively interpret this propagation as 

\begin{quote}
``Relevance $\hat{R}_k$'' should be redistributed to the lower-
layer neurons $\{a_j\}_j$ in proportion to their excitatory effect on $a_k$. ``Counter-relevance'' $\check{R}_k $ should be redistributed to the lower-layer neurons $\{a_j\}_j$ in proportion to their inhibitory effect on $a_j$
	- Section 5.1 \cite{MontavonMethodsInterpretingUnderstanding2017}
\end{quote} 

Moreover, LRP also provides \textit{conservation property} in which total relevance quantities are conserved during propagating $f(\patvector{x})$ back to $\x$. Formally, it is 
\begin{align}
	\sum_{i} R_i =  \cdots =	\sum_{j} R_j = \sum_{k} R_k = \cdots = f(\patvector{x})
\end{align}


Nonetheless, choices of $\alpha,\beta$ is still a question for LRP.  In particular, \cite{MontavonMethodsInterpretingUnderstanding2017, BinderLayerWiseRelevancePropagation2016} demonstrated that the influence of the values to the quality of explanation are depend on network architecture. For example, \cite{MontavonMethodsInterpretingUnderstanding2017} observed that LRP-${\alpha_1, \beta_0}$ works well for deep architectures, such as GoogleNet\cite{SzegedyGoingDeeperConvolutions2014}, while LRP-${\alpha_2, \beta_1}$ is better for shallower architectures, such as BVLC CaffeNet\cite{JiaCaffeConvolutionalArchitecture2014}.


%However, it seems that there is no clear relationship between values of $\alpha,\beta$ and the structure of the heatmap.  demonstrate that the values are depend on the architecture of the network. In particular, 
%




\subsection{Simple Taylor Decomposition}
As the name suggested,  the method decomposes $f(\patvector{x})$ using Taylor expansion around a root point $\tilde{\x}$. The relevance scores $R_i(\x)$ are interpreted as the first order terms of the series. It is formally defined as 

% into terms of relevance scores $R_i$ via Taylor decomposition. Formally, 

\begin{align} \label{eq:simple_taylor_expansion}
	f(\patvector{x}) 	&= f(\tilde{\patvector{x}}) + \sum_{i} \underbrace{\frac{\partial f }{ \partial x_i } \Bigg|_{x_i = \tilde{x}_i}  ( x_i - \tilde{x}_i ) }_{R_i} + \zeta, 
\end{align}
where $\zeta$ are the second and higher order terms that are not considered here. The root point $\tilde{\x}$ can be found via the optimization below 
\begin{align}
\underset{\xi \in \mathcal{X} }{\text{min}}  || \xi - \patvector{x} ||^2 \hspace{2cm}  \text{such that}\  f(\xi) = 0,
\end{align}

where $\mathcal{X}$ represents the input distribution. This optimization is time consuming  and $\xi$ might potentially diverge from $\patvector{x}$ leading to non informative $R_i$. However, for neural networks using piecewise linear activations, such as ReLU,  $\tilde{\x}$ can be computed analytically. In particular, with assumptions of $\sigma(tx) = t\sigma(x) ,\forall t \ge 0$ and no use of bias, \cite{MontavonMethodsInterpretingUnderstanding2017} argued that $\tilde{\patvector{x}}$ can be found in  approximately the same flat region as $\patvector{x}$, $\tilde{\patvector{x}} = \underset{\epsilon \rightarrow 0 }{\lim} \epsilon \patvector{x}$, yielding

\begin{align}
	\frac{\partial f(\patvector{x})}{\partial x_i}\bigg|_{\patvector{x}={\tilde{\patvector{x}}}} = \frac{\partial f(\patvector{x})}{\partial x_i}\bigg|_{\patvector{x}={\patvector{x}}} 
\end{align}

%demonstrate that  neural networks whose activations $\sigma(x)$ are piecewise linear functions  with , for example a deep Rectified Linear Unit (ReLU) network without biases, 
%
As a result, (\ref{eq:simple_taylor_expansion}) can be simplified to :
\begin{align}
	f(\patvector{x}) &= \sum_{i} \frac{\partial f(\patvector{x})}{\partial x_i}\bigg|_{\patvector{x}={\patvector{x}}}  x_i \\
	R_i &= \frac{\partial f(\patvector{x})}{\partial x_i}\bigg|_{\patvector{x}={\patvector{x}}}  x_i \label{eq:simple_r_i}
\end{align} 

This derivation shows a relationship between SA and simple Taylor decomposition. Specifically, $x_i$ will have high relevance score if $x_i$ highly activates and its variation positively affects $f(x)$ and vice versa.


\subsection{Deep Taylor Decomposition}
Deep Taylor decomposition(DTD) is another local explanation technique that rely on Taylor expansion. Unlike simple Taylor decomposition, DTD instead decomposes $R_k$ into $R_j$, which are the first order terms of the series. \cite{MontavonExplainingnonlinearclassification2017} proposed the method to explain decisions of neural networks with piece-wise linear activations. Similar to LRP, DTD successively decomposes relevances at the last layer $R_k$ and propagates the quantities to $R_j$ in the previous layer until the relevance of the input$R_i(\x)$. Formally, $R_k$ is decomposed as follows :


% In fact, it can be shown that LRP's propagation rule is equivalent to one of DT's rules.


 \begin{align} \label{eq:tl_rj}
 R_k = R_k \bigg|_{ \tilde{\patvector{a}}_j } + \sum_{ j } 	\frac{\partial  R_k }{ \partial a_j } \bigg|_{ a_j = \tilde{a}_j } ( a_j - \tilde{a}_j ) + \zeta_k
 \end{align}

Let's further Assume that there exists a root point $\tilde{\patvector{a}}_j$ such that $R_k = 0$, and the second and higher terms $\zeta_k = 0 $. Then, (\ref{eq:tl_rj}) can be reduced to
\begin{align} \label{eq:R_k_sum}
 R_k = \sum_{ j } \underbrace{	\frac{\partial  R_k }{ \partial a_j } \bigg|_{ a_j = \tilde{a}_j }  ( a_j - \tilde{a}_j ) }_{ R_{j \leftarrow k } }
\end{align}

As the relevance propagated back,  $R_j$ is $\sum_{k} R_{j\leftarrow k}$ from neuron $k$ in the successive layer that neuron $j$ contributes to. Hence, DTD also has \textit{conservation property}, 
%Combining Equation \ref{eq:R_k_sum} and \ref{eq:R_j_sum} yields :
\begin{align} 
	R_j &= \sum_{k} R_{j\leftarrow k} \label{eq:rj_from_rk} \\
\sum_{j}	R_j &= \sum_{j} \sum_{k} R_{j\leftarrow k}\\
\sum_{j}	R_j &= \sum_{k} \sum_{j} R_{j\leftarrow k} \\
\sum_{j}	R_j &= \sum_{k}  R_{k} \\
\sum_i 	R_{i} = 	\dots = \sum_j R_{j} &= \sum_k R_{k} = \dots =  f(\patvector{x}) \label{eq:rj_equal_rk}
\end{align}

%Demonstrated by \cite{MontavonExplainingnonlinearclassification2017}, Equation \ref{eq:rj_equal_rk} holds for all $j, k$ and all subsequent layers. Hence, this results in  conservation property which guarantee that no relevance loss during the propagations.
%\begin{align}
%\sum_i 	R_{i} = 	\dots = \sum_j R_{j} = \sum_k R_{k} = \dots =  f(\patvector{x})
%\end{align}
 
 
%\begin{figure}[h]
%\centering
%\includegraphics[width=0.8\textwidth]{sketch/deep_tayloy_decomposition_toy}
%\caption{A Simple network}
%\label{fig:deep_tayloy_decomposition_toy}
%\end{figure}

\subsubsection{Finding the root point}
Consider a neural network whose $R_k$ is computed by
% is based on activations of $a_j$ in the previous layer and  :
\begin{align}\label{eq:r_k_deep_taylor}
R_k = \text{max}\ \bigg(0, \sum_{j} a_j w_{jk}  + b_k \bigg),
\end{align}
where $a_j$ are activations of neurons in previous layer and  $b_k \le 0 $.

To find the root point $\tilde{\patvector{a}}_j$, one can see that  there are  2 cases to be considered, namely when $R_k = 0$ and $R_k > 0$. For $R_k=0$,  $\patvector{a}_j$ is already the root point. For the latter, the root point can be found by performing  line search in  some direction $\patvector{v}_j$ with magnitude $t$.
\begin{align}\label{eq:root_aj_aj}
\tilde{a}_j = a_j - t v_j,
\end{align}
More precisely, the root point is the intersection point between (\ref{eq:r_k_deep_taylor}) and (\ref{eq:root_aj_aj}) where $R_k=0$. Hence,
\begin{align}
  0 &= 	\sum_{j} (a_j - t v_j) w_{jk}  + b_k\\
%  0 &= \sum_{j} a_j w_{jk} - t v_j w_{jk}  + b_k\\
  t \sum_{j} v_j w_{jk} &= R_k \\
  t &= \frac{R_k}{\sum_{j} v_j w_{jk}}
\end{align}

Therefore, $R_j$ can be then calculated by
\begin{align}
R_j &= \sum_k	\frac{\partial  R_k }{ \partial a_j } \bigg|_{ a_j - \tilde{a}_j }  ( a_j - \tilde{a}_j ) \\
&=	\sum_k w_{jk} tv_j \\
&=	\sum_k \frac{ v_j w_{jk}   }{\sum_{j} v_j w_{jk}}  R_k
\end{align}

\begin{figure}[!hbt]
\centering
\includegraphics[width=0.4\textwidth]{sketch/invalid_root_point_example}
\caption{Function's view of $R_k$  and root point candidates}
\label{fig:root_point_illus}
\end{figure}



The last step is to find $\boldsymbol{v}_j$ such that $\tilde{\patvector{a}}_j$ is the closest point to the line $R_k=0$ and also in the same domain as $\patvector{a}_j$. \addfigure{\ref{fig:root_point_illus}} visualizes the intuition. Here, if $a_j \in \mathbb{R}^+$, then $\tilde{a}_j$ must be also in $\mathbb{R}^+$. Therefore, $\boldsymbol{v}_j$ needs to be separatedly derived for each possible domain.
% of $\patvector{a}_j$.

\subsubsection{Case $a_j \in \mathbb{R}$ : $w^2$-rule}

Trivially, the search direction $v_j$ is just the direction of derivative,
\begin{align}
	v_j &= \frac{\partial R_k}{ \partial a_j } \\
	&= w_{jk}
\end{align}

Hence, 
\begin{align}
	R_j &=	\sum_k \frac{ w_{jk}^2  }{\sum_{j} w_{jk}^2}  R_k
\end{align}

\subsubsection{Case $a_j \in \mathbb{R}^+$ : $z^+$-rule}
\begin{figure}[!htb]
\centering
\subfloat[]{%
       \includegraphics[width=0.4\textwidth]{sketch/zplus_rule_case_1}
     }
%     \hfill
     \subfloat[]{%
       \includegraphics[width=0.4\textwidth]{sketch/zplus_rule_case_2}
     }
\caption{Function's view of $R_k$ and the root point from $z^+$-rule}
\label{fig:zplus_rule_cases}
\end{figure}
In this case, the root point is on the line segment $( a_j \mathbbm{1}[ w_{jk}  < 0 ], a_j )$. In particular, as shown on \addfigure{\ref{fig:zplus_rule_cases}}, $R_k$ has the root point at $a_j \mathbbm{1}[w_{jk}  < 0 ]$, because 
\begin{align}	
R_k &= \max \bigg( \sum_j a_j w_{jk } + b_k, 0 \bigg) \\
&=  \max \bigg( \sum_j a_j \mathbbm{1}[ w_{jk}  < 0 ] w_{jk} + b_k, 0 \bigg) \\
&=  \max \bigg(\sum_j a_j  w_{jk}^- + b_k , 0\bigg) \\
&= 0 \label{eq:zplus_rk_zero}
\end{align}
The last step uses the assumptions that $a_j \in R^+$ and $b_k \le 0$. Hence, the search direction is 
\begin{align}
	v_j &= a_j - a_j \mathbbm{1}[ w_{jk}  < 0 ] \\
	&= a_j \mathbbm{1}[ w_{jk}  \ge 0 ]
\end{align}

Therefore, 
\begin{align}
		R_j &=	\sum_k \frac{ w_{jk} a_j \mathbbm{1}[w_{jk}  \ge 0 ]  }{\sum_{j} w_{jk} a_j \mathbbm{1}[ w_{jk}  \ge 0 ]}  R_k\\
		&=	\sum_k  \frac{ a_j  w_{jk}^+   }{\sum_{j}  a_j w_{jk}^+  }  R_k
\end{align}

In fact, this propagation rule is equivalent to LRP-$\alpha_1\beta_0$. 


\subsubsection{Case $a_j \in [l_j , h_j]$ where $l_j \le 0 < h_j $ : $z^\beta$-rule}

\begin{figure}[!htb]
\centering
\subfloat[]{%
       \includegraphics[width=0.4\textwidth]{sketch/zbeta_rule_case_1}
     }
%     \hfill
     \subfloat[]{%
       \includegraphics[width=0.4\textwidth]{sketch/zbeta_rule_case_2}
     }
\caption{Function's view of $R_k$ and the root point from $z^\beta$-rule with $-1 < a_j < 1$ }
\label{fig:zbeta_rule_cases}
\end{figure}
This propagation rule is considered for the first layer whose input's value is bounded, for example pixel intensity. As shown in \addfigure{\ref{fig:zbeta_rule_cases}}, the root point is on the line segment $( l_j \mathbbm{1}[ w_{jk}  \ge 0 ]  + h_j \mathbbm{1}[ w_{jk}  \le 0 ]  , a_j ) $. More precisely, the root point is at $l_j \mathbbm{1}[ w_{jk}  \ge 0 ]  + h_j \mathbbm{1}[ w_{jk}  \le 0 ]$ because
\begin{align}
R_k &= \max \bigg( \sum_j a_j w_{jk } + b_k, 0 \bigg) \\
&=\max \bigg( \sum_j (l_j \mathbbm{1}[ w_{jk}  \ge 0 ]  + h_j \mathbbm{1}[ w_{jk}  \le 0 ] ) w_{jk } + b_k, 0 \bigg) \\
&= \max \bigg( \sum_j l_j w_{jk }^+   + h_j w_{jk }^-  + b_k, 0 \bigg) \\
&= 0
\end{align}

Hence,  the search direction is 
\begin{align}
	v_j &= a_j - \tilde{a}_j \\
	&=a_j  - l_j \mathbbm{1}[ w_{jk}  \ge 0 ]  - h_j \mathbbm{1}[ w_{jk}  \le 0 ]
\end{align}

Therefore, 
\begin{align}
		R_j &=	\sum_k \frac{ w_{jk}  (a_j  - l_j \mathbbm{1}[ w_{jk}  \ge 0 ]  - h_j \mathbbm{1}[ w_{jk}  \le 0 ]) }{\sum_{j} w_{jk}  (a_j  - l_j \mathbbm{1}[ w_{jk}  \ge 0 ] - h_j \mathbbm{1}[ w_{jk}  \le 0 ]) }  R_k\\
		&=	\sum_k  \frac{ a_j  w_{jk} - l_j w_{jk}^- - h_j w_{jk}^+  }{\sum_{j}   a_j  w_{jk} - l_j w_{jk}^- - h_j w_{jk}^+  -}  R_k
\end{align}





%In summary, DTD is a theory for explaining nonlinear compuations of neural network through decomposing relevance score between successive layers. Its propagation rules ensures conservation property. Given the rules above, the relevance scores can be propagated using LRP Algorithm \ref{algo:lrp}.
%
%Lastly, as DTD provides more general propagation rules than the $\alpha\beta$-rule from LRP, I will use DTD and LRP interchangeably throughout the thesis. In particular, it will be mentioned explicitly if $\alpha\beta$ is being used, otherwise the rule is a DTD's rule.  

Table \ref{tab:lrp_deep_taylor_rules} concludes the details  when each DTD rule should be applied. With this knowledge, $f(\x)$ can be then propagated to the input using LRP Algorithm \ref{algo:lrp}.


\renewcommand{\arraystretch}{1}
\begin{table}[h]
\centering
\begin{tabular}{|l|l|}
\hline
\multicolumn{1}{|c|}{Rule and input domain} & \multicolumn{1}{c|}{Formula} \\ \hline
$w^2$-rule : Real values,  $a_j \in \mathbb{R}$ & \parbox{1cm}{
	\begin{align*}
		R_j =	\sum_k \frac{ w_{jk}^2  }{\sum_{j} w_{jk}^2}  R_k  	
    \end{align*}}
 \\ \hline
% &                                \\ \hline
$z^+$-rule : ReLU activations, $a_j \in \mathbb{R}^+$    & \parbox{1cm}{\begin{align*}
R_j = \sum_k  \frac{ a_j  w_{jk}^+   }{\sum_{j}  a_j w_{jk}^+  }  R_k	
\end{align*}} \\ \hline
$z^\beta$-rule : Pixel Intensities, $ a_j \in [l_j , h_j]$ where $l_j \le 0 < h_j $  & \parbox{1cm}{\begin{align*}
R_j = \sum_k  \frac{ a_j  w_{jk} - l_j w_{jk}^- - h_j w_{jk}^+  }{\sum_{j}   a_j  w_{jk} - l_j w_{jk}^- - h_j w_{jk}^+  -}  R_k	
\end{align*}}
               \\ \hline
\end{tabular}
\caption{Relevance propagation rules of deep Taylor decomposition. }
\label{tab:lrp_deep_taylor_rules}
\end{table}
\renewcommand{\arraystretch}{1}


\begin{figure}[!htb]
\centering
\includegraphics[width=\textwidth]{sketch/lenet_heatmaps}
\patcaption{Relevance heatmaps produced by different explanation methods explaining decisions of LeNet-5.}{\heatmapscaleexplain }
\label{fig:lenet_heatmaps}
\end{figure}

In this section, we have described intuition and details of several local explanation methods. \addfigure{\ref{fig:lenet_heatmaps}} shows  examples of relevance heatmaps from those methods explaining classification decisions of LeNet-5\cite{LeCunGradientBasedLearningApplied2001}. The network was trained on 100 epochs with batch size 50 and dropout probability at 0.2. It achieves accuracy at 99.21\% for MNIST and 87.90\% for FashionMNIST. The figure also well presents characteristics of explanation from each method. In particular, one can observe that simple Taylor decomposition provides the most noisy and non informative explanations, while the ones from SA also look noisy but some structures of input can be seen. GB produces smoother explanations than SA. On the other hand, explanations from deep Taylor decomposition(DTD) and LRP are more diffuse and smoothest : Both methods have similar relevance heatmaps when $\beta$ is small. Given this result, we are going to consider only SA, GB, DTD and $\lrpp$ in experiments.

	
	
	\chapter{Experiments}
\label{cha:chapter4}


\section{General Setting}\label{sec:setup}
 
 We use \textit{Adaptive Moment Estimation(Adam)}\cite{KingmaAdamMethodStochastic2014} to train models and initialize weights $w_{ij} \in \patmatrix{W}$ and biases $b_{j} \in \patvector{b}$ as follows
\begin{align*}
	w_{ij} &\sim \Psi( \mu, \sigma, [-2\sigma, 2\sigma]) \\
	b_{j} &= \ln(e^{0.01} - 1)
\end{align*}
where $\Psi(\cdot)$ denotes \textit{truncated normal distribution} where $\mathbb{P}(|w_{ij}| > 2\sigma) = 0$. Precisely, we use $\mu=0$ and $\sigma = 1/\sqrt{|\boldsymbol{a}|}$ where $|\boldsymbol{a}|$ is the number of neurons in previous layer.

The activations of neurons in each layer $j$, denoted as $\patvector{a}^{(j)}$, are computed using 
\begin{align*}
\patvector{h}^{(j)}  &=  	(\patmatrix{W}_{ i \rightarrow j })^T \patvector{a}^{(i)} - \sigma_{s}(\patvector{b_j}) \\
	\patvector{a}^{(j)}  &=  	\sigma_{r} (	\patvector{h}^{(j)} )
\end{align*}

where $\sigma_r(\cdot)$ and $\sigma_s(\cdot)$ are \textit{ReLU} and \textit{softplus} function respectively and applied element-wise to elements in the vectors.

 \begin{figure}[!hbt]
\centering
\includegraphics[width=0.5\textwidth]{relu_softplus}
\caption{ReLU and Softplus function} 
\label{fig:relu_softplus}
\end{figure}


The reason of using softplus function for the bias term is due to the non positive bias assumption of DTD. Secondly, the continuity of softplus function is another reason. This property allows the bias term to be more flexibly adjusted through back-propagation than using ReLU. With this setting, the initial value of bias term  $\sigma_{s}(b_j)$ is then 0.01.


\begin{table}[!htb]
\centering
\begin{tabular}{l|r}
\textbf{Hyperparameter} & \multicolumn{1}{l}{\textbf{Value}} \\ \hline
Optimizer               & Adam                               \\
Epoch     & 100                                \\
Dropout Probability     & 0.2                               \\
Batch size              & 50                                
\end{tabular}
\caption{Summary of hyperparameter.}
\label{tab:hyper_summary}
\end{table}


\begin{table}[h]
\centering
\begin{tabular}{ll}
\multicolumn{1}{l|}{\textbf{Dataset}} & \textbf{Minimum Accuracy} \\ \hline
\multicolumn{1}{l|}{MNIST}            & \multicolumn{1}{r}{98.00\%}  \\
\multicolumn{1}{l|}{FashionMNIST}    & \multicolumn{1}{r}{85.00\%}  \\
%\multicolumn{1}{l|}{UFI Cropped}                                       &                         \dots
\end{tabular}
\caption{Minimum classification accuracy for models to be considered.}
\label{tab:min_acc}
\end{table}



Dropout technique \cite{SrivastavaDropoutSimpleWay2014} is applied to activations of every fully-connected layer, unless stated otherwise. We use the dropout probability at 0.2. We train models with batch size 50 for 100 epochs. Table \ref{tab:hyper_summary} summaries the setting of hyperparameters. On the other hand, learning rate is not globally fixed and left adjustable per architecture: we use the value between 0.0001 and 0.0005. Based on literature surveys, Table \ref{tab:min_acc} shows minimum accuracy of each dataset. Models considered in the following experiments need to satisfy this requirement.  Numbers of neurons in each layer were carefully chosen such that every architecture has similar number of trainable variables. More precise configuration will be discussed separately in each experiment. 

Consider RNN with parameters $\boldsymbol{\theta} = \{ \patmatrix{W}, \boldsymbol{b} \}$. Assume that $g_r$ and $g_{f}$ are functions that the RNN uses to compute recurrent input $\patvector{r}_{t+1}$ and $f(\x)$ respectively. For a classification problem with $K$ classes, the calculations can be  roughly summarized as follows
 \begin{align}
 	\patvector{r}_{t+1} &= g_r(\patvector{\theta}, \patvector{x_t}, \patvector{r_t}) \\
 	 &\ \ \vdots\\
f(\x) &= g_{f}(\patvector{\theta}, \patvector{x}_{T},  \patvector{r}_{T}) \\
 	\patvector{\hat{y}} &= \text{softmax}(f(\x)),
 \end{align}
 where $t \in \{1, \dots, T\}$, $(\x_t \in \x)_1^{T}$ are input corresponding to  step $t$, $\patvector{r}_0 = \patvector{0}$, and $\patvector{\hat{y}} \in \mathbb{R}^K$ are the vector of class probabilities. To compute explanation or relevance heatmap of $\x$, denoted as $R(\x)$, we take $z^* \in f(\x)$ that is corresponding to the true target class, instead of the predicted class.  Because DTD and LRP method are primarily  based on distributing positive relevance, we also introduce a constant input, whose value is zero, to the softmax function to force the network building positive relevance, $z^* \in \mathbb{R}^+$. Mathematically, this constant does not affect the training procedure.

Our implementation is written in Python and based on TensorFlow\cite{AbadiTensorFlowLargeScaleMachine2016}. It is publicly available on Github\footnote{\url{https://github.com/heytitle/thesis-designing-recurrent-neural-networks-for-explainability/releases/tag/release-final}}.  We run our experiments either on a GeForce GTX 1080 provided by TUB ML group or AWS's p2.xlarge\footnote{\url{https://aws.amazon.com/ec2/instance-types/p2/}} instance. With this setting, each model approximately takes 2 hour to train.


 
 % 
%Traditionally, number of neurons in each layer ($n^{(l)}$) is  another hyperparameter that we can adjust. However, as the goal is to compare relevance heatmaps from different architectures, those numbers are fixed and chosen in such a way that total number of variables in each architecture are equivalent. \addfigure{\ref{fig:neuron_numbers}} illustrates the details of the settings.
%
%\begin{figure}[!htb]
%\centering
%\includegraphics[width=\textwidth]{sketch/neuron_numbers}
%\caption{Number of neurons in each layer for each cell architecture}
%\label{fig:neuron_numbers}
%\end{figure}
%
%
%\begin{itemize}
%	\item \textbf{Shallow Cell} 
%$$\{ n^{(4)}\} = \{ 768 \}$$
%	\item \textbf{Deep Cell} 
%$$\{ n^{(1')}, n^{(4)}, n^{(4')} \} = \{ 512, 256, 64 \}$$
%	\item \textbf{DeepV2 Cell} 
%$$\{ n^{(1')}, n^{(1")}, n^{(4)}, n^{(4')} \} = \{ 512, 256, 128, 64 \}$$
%	\item \textbf{ConvDeep Cell} : 
%\begin{align*}
%	\{ n^{(C1)}, n^{(P1)} \} &= \{ CONV(5\text{x}5, 24), POOL(2\text{x}2) \} \\
%		\{ n^{(C2)}, n^{(P2)} \} &= \{ CONV(3\text{x}3, 48), POOL(2\text{x}2) \} \\
%			\{  n^{(4)}, n^{(4')} \} &= \{ 256, 128 \}
%\end{align*}
%where $CONV(x,y)$ is a convolutional operator with $y$ filters whose kernel size is $\mathbb{R}^{x}$. Similarly, $POOL(x)$ is a pooling operator  with kernel size $\mathbb{R}^{x}$.
%
%
%\end{itemize}
%
%Noting that, $n^{(5)}$ is set at 128 for all architectures and 0 when the sequence length of the problem is 1. $n^{(6)}$ is equal to the number of categories of a problem, for example $n^{(6)} = 10 $ MNIST. Table \ref{tab:variable_architecture} shows the total numbers of variables in details.

%\renewcommand{\arraystretch}{1.2}
%\begin{table}[h]
%\centering
%\begin{tabular}{l|c|c|c|}
%\cline{2-4}
%                                                 & \multicolumn{3}{c|}{\textbf{Sequence Length}} \\ \hline
%\multicolumn{1}{|l|}{\textbf{Cell Architecture}} & 1         & 4         & 7         \\ \hline
%\multicolumn{1}{|l|}{\rnncell{Shallow}}                    & 610570    & 355722    & 291210      \\ \hline
%\multicolumn{1}{|l|}{\rnncell{Deep}}                       & 550346    & 314954    & 271946      \\ \hline
%%\multicolumn{1}{|l|}{\rnncell{DeepV2}}                    & 575050    & 306890    & 263882      \\ \hline
%%\multicolumn{1}{|l|}{\rnncell{ConvDeep}}                   & 647594    & 283178    & 197162      \\ \hline
%\end{tabular}
%\caption{Total variables in each architecture and sequence length}
%\label{tab:variable_architecture}
%\end{table}


 

%\clearpage
\section{Experiment 1 : Sequence Classification}
\label{sec:exp1}

\subsection{Problem Formulation}
We consider this experiment as a preliminary study. Here, we constructed an artificial classification problem in which each image sample $\x$ is column-wise split into a sequence of non-overlapping $(\x_t)_{t=1}^{T}$. The RNN classifier needs to summarize information from the sequence $(\x_t)_{t=1}^{T}$ to determine what is the class of $\x$.   Using image allows us to conveniently inspect produced explanations.

\addfigure{\ref{fig:artificial_problem}} illustrates the setting. As shown in the figure, a MNIST sample $ \patvector{x} \in \mathbb{R}^{28,28}$ is divided to a sequence of $( \patvector{x}_t \in   \mathbb{R}^{28,7} )_{t=1} ^ 4$. At time step $t$, $\patvector{x}_t$ is presented to the RNN classifier yielding recurrent input $\patvector{r}_{t+1}$ for the next step. For the last step $T = 4$, the RNN classifier computes $f(\x) \in \mathbb{R}^{10}$ and the class probabilities accordingly.


 \begin{figure}[!hbt]
		\centering
		\includegraphics[width=\textwidth]{sketch/artificial_problem_with_rel}
		\caption{RNN Sequence classifier and decision explanation} 
		\label{fig:artificial_problem}
\end{figure}


\begin{figure}[!htb]
\centering

\subfloat[Shallow architecture\label{fig:shallow_arch}]{%
       \includegraphics[width=0.48\textwidth]{sketch/shallow_arch}
     }
     \hfill
     \subfloat[Deep architecture\label{fig:deep_arch}]{%
       \includegraphics[width=0.48\textwidth]{sketch/deep_arch}
     }
\patcaption{Shallow and Deep architecture}{with number of neurons at each layer depicted.}
\end{figure}

In this experiment, we are going to  consider 2 RNN architectures, namely

\begin{enumerate}
	\item \textbf{Shallow architecture} \\
		As shown in \addfigure{\ref{fig:shallow_arch}}, the \rnncell{Shallow} architecture first concatenates  input $\patvector{x}_t$  and recurrent input $\patvector{r}_t$ at layer \circled{3} as one vector before computing activations of layer \circled{4}, denoted as $\patvector{a}_t^{(4)}$. Then,  the next recurrent input $\patvector{r}_{t+1}$ \circled{5}	 is derived from $\patvector{a}_t^{(4)}$. In the last step $T$, $f(\x)$ is computed from $\patvector{a}^{(4)}_{T}$ and applied to softmax function to compute class probabilities $\patvector{\hat{y}}$. 
		
Because activations coming to \circled{4} are from different domains, $\x_t \in [-1, 1]$ and $\patvector{r}_t \in [0, \infty) $, a special propagation rule is needed in order to apply DTD.  Denoting $i$ and $j$ the pixels in $\x_t$ and activations in $\patvector{r}_t$ respectively and $k$ the corresponding neuron in \circled{4}, the propagation rule is defined as 
		
\begin{align}
	R_{t, i} &= \sum_k \frac{(x_{t, i} w_{ik} - l_i w_{ik}^+ - h_i w_{ik}^-) R_{t,k}}{z_{t,k}} \\	
     R_{t, j} &= \sum_k \frac{r_{t, j} w_{jk}^+ R_{t,k}}{z_{t,k}}
\end{align}
where $z_{t,k} = \sum_j r_{t,j} w_{jk}^+ + \sum_i x_{t,i} w_{ik} - l_i w_{ik}^+ - h_i w_{ik}^-$ is the normalization term.

	\item \textbf{Deep architecture} \\
		\addfigure{\ref{fig:deep_arch}} illustrates the configuration of this architecture. Unlike the Shallow architecture, the Deep architecture has 2 more layers in the beginning, namely \circled{1'} and \circled{4'}.  The improvement would allow \circled{1'} to properly learn representations from input and \circled{4} to efficiently combine information from the past and current input. Hence, this then enables \circled{4'} to compute more fine-grained decision probabilities leading to better overall quality of explanation.
\end{enumerate}

\renewcommand{\arraystretch}{1.5}
\begin{table}[h]
\centering
\begin{tabular}{cc|c|c|}
\cline{3-4}
& & \multicolumn{2}{c|}{\textbf{No. trainable variables}}                                                                \\ \hline
\multicolumn{1}{|c|}{\textbf{T}}               & \multicolumn{1}{c|}{\textbf{Dim. of $\x_t$}} & \multicolumn{1}{c|}{\textbf{Shallow}} & \multicolumn{1}{c|}{\textbf{Deep}}  \\ \hline
\multicolumn{1}{|c|}{1} & $\mathbb{R}^{28,28}$ & 269,322  &  271,338 \\
\multicolumn{1}{|c|}{4} & $\mathbb{R}^{28,7}$ & 184,330 & 153,578 \\
\multicolumn{1}{|c|}{7} & $\mathbb{R}^{28,4}$ & 162,826 & 132,074 \\ \hline

\end{tabular}
\caption{Dimensions of $\patvector{x}_t$ and number of trainable variables in each architecture on different sequence length $T=\{1, 4, 7\}$.}
\label{tab:seq-length}
\end{table}
\renewcommand{\arraystretch}{1}




We experimented with MNIST and FashionMNIST using sequence length $T = \{1, 4, 7\}$.  Table \ref{tab:seq-length} shows dimensions of $\patvector{x}_t$ for different sequence length as well as number of trainable variable in each architecture.

To simplify the writing, we are going to use \textit{\rnncellseq{ARCHITECTURE}{T}} convention to denote a RNN with \textit{ARCHITECTURE} trained on the sequence length \textit{T}. For example, \rnncellseq{Deep}{7} refers to the Deep architecture trained on $(\x_t \in \mathbb{R}^{28,4} )_{t=1}^{7}$.

\subsection{Result}
\label{sec:exp1_result}

\renewcommand{\arraystretch}{1.5}
\begin{table}[]
\centering
\begin{tabular}{cc|c|c|c|}
\cline{2-5}
& \multicolumn{2}{|c|}{\textbf{MNIST}} & \multicolumn{2}{|c|}{\textbf{FashionMNIST}} \\ \hline
\multicolumn{1}{|c|}{$T$}   & \multicolumn{1}{c|}{\textbf{Shallow}} & \multicolumn{1}{c|}{\textbf{Deep}} & \multicolumn{1}{c|}{\textbf{Shallow}} & \multicolumn{1}{c|}{\textbf{Deep}} \\ \hline
\multicolumn{1}{|c|}{1} & 98.11\%   & 98.22\% & 87.93\%  & 89.14\%                           \\
\multicolumn{1}{|c|}{4} & 98.56\% & 98.63\%  & 89.04\%  & 89.43\%                            \\
\multicolumn{1}{|c|}{7} & 98.66\%  & 98.68\% & 89.28\%  & 88.96\%  \\ \hline
\end{tabular}
\caption{Sequence classification accuracy from Shallow and Deep architecture trained with different sequence length.}
\label{tab:mnist_model_acc}
\end{table}
\renewcommand{\arraystretch}{1}



 \begin{figure}[!htb]
\centering
\includegraphics[width=0.8\textwidth]{sketch/mnist_experiment}
\patcaption{Relevance heatmaps from different explanation techniques applied to Shallow and Deep architecture trained on MNIST with different sequence lengths.}{\heatmapscaleexplain }
\label{fig:mnist_experiment}
\end{figure}


Table \ref{tab:mnist_model_acc} summaries accuracy of the trained models. Both Shallow and Deep architecture have comparable accuracy, hence their explanations can also be compared. \addfigure{\ref{fig:mnist_experiment}} shows relevance heatmaps from Shallow and Deep architecture trained on MNIST.  We can observe similar characteristics of each explanation technique as in \addfigure{\ref{fig:lenet_heatmaps}}. In particular, SA and GB explanations are sparse, while the ones from DTD and $\lrpp$ are more diffuse throughout $\x$. 

\rnncellseq{Shallow}{1}  and \rnncellseq{Deep}{1} have similar relevance heatmaps regard less of explanation methods.  As the sequence length increased, \rnncellseq{Shallow}{4,7} and  \rnncellseq{Deep}{4,7} start producing significantly different relevance heatmaps when explained by DTD and $\lrpp$.  In particular,  \rnncellseq{Shallow}{4,7} 's explanations are mainly concentrated on the right, which associate to input of last time steps, while  \rnncellseq{Deep}{4,7}'s ones are proportionally  highlighted around content area of $\x$. We do not see this effect from SA and GB.


 \begin{figure}[!htb]
\centering
\includegraphics[width=0.8\textwidth]{sketch/fashion_mnist_experiment}
\patcaption{Relevance heatmaps from different explanation techniques applied to  Shallow and Deep architecture trained on FashionMNIST with different sequence lengths.}{\heatmapscaleexplain}
\label{fig:fashion_mnist_experiment}
\end{figure}

Relevance heatmaps of Shallow and Deep architecture trained on  FashionMNIST  are shown on \addfigure{\ref{fig:fashion_mnist_experiment}}. Similar to the ones from MNIST, we do not see any remarkable difference on SA and GB heatmaps of the two architectures although \rnncellseq{Deep}{4,7} produces slightly more sparse heatmaps than \rnncellseq{Shallow}{4,7}. However, the wrong concentration issue of DTD and LRP seems to appear on both \rnncellseq{Shallow}{4,7}'s and \rnncellseq{Deep}{4,7}'s heatmaps. Nevertheless, we can still observe proper highlight from Deep architecture on some samples. For example, the trouser sample, we can see  that \rnncellseq{Deep}{4,7} architecture manage to distribute high relevance scores to area of the trouser. 

Similar structures of FashionMNIST samples might be one of the reasons why Deep architecture is not able to distribute relevance scores to earlier steps as in MNIST cases. Consider \textit{Shoe} and \textit{Ankle Boot} samples in \addfigure{\ref{fig:fashion_mnist_samples}}. One can see that  their front part are similar and only the heel part that determines the difference between the two categories. This evidence suggests that  more robust feature extractor layer, such as convolution and pooling layer, might be more suitable than the fully-connected one.


 \begin{figure}[!htb]
\centering
\includegraphics[width=0.8\textwidth]{sketch/class_1_comparison}
\patcaption{Relevance heatmaps of MNIST \textit{Class 1} and FashionMNIST \textit{Class Trouser} samples from \rnncellseq{Shallow}{7} and \rnncellseq{Deep}{7} explained by DTD and $\lrpp$.}{\heatmapscaleexplain }
\label{fig:class_1_comparison}
\end{figure}

\addfigure{\ref{fig:class_1_comparison}} presents explanations of MNIST \textit{Class 1} and FashionMNIST \textit{Class Trouser} samples. These samples were chosen to emphasize the impact of RNN architecture on DTD and LRP explanation. As can be seen from the figure, these samples have $\x_{t'}$ containing actual content  primarily locating at the center, or middle of the sequence. Hence, relevance heatmaps should be highlighted at $\x_{t'}$ and possibly its neighbors.  As expected, we can see \rnncellseq{Deep}{7} produces sound explanations in which the heatmaps have high intensity value where $\x_{t'}$ approximately locate, while \rnncellseq{Shallow}{7} mainly assigns relevance quantities to $\x_{t}$ for $t \approx T$. 

\addfigure{\ref{fig:exp1_dist_plot}} further shows a quantitive evidence of this wrong propagation issue of DTD and $\lrpp$. Here, distributions of relevance scores derived from the methods on \rnncellseq{Shallow}{7} and \rnncellseq{Deep}{7} are plotted across time step $t = \{ 1, \dots, 7 \}$. The distributions are computed from all test samples in MNIST \textit{Class 1} and FashionMNIST \textit{Class Trouser} respectively. The plots also include distribution of pixel  values.  We can see that the relevance distributions from \rnncellseq{Deep}{7} align with the data distributions, while  the distributions from \rnncellseq{Shallow}{7} ones diverge with significant margin. Approximately,  one can see that \rnncellseq{Shallow}{7} distributes more than 90\% of relevance scores to the last 3 steps, namely $\x_5$, $\x_6$ and $\x_7$.


 \begin{figure}[!htb]
\centering
\includegraphics[width=\textwidth]{sketch/exp1_dist_plot}
\caption{Distribution of pixel intensity, relevance quantities from \rnncellseq{Shallow}{7} and \rnncellseq{Deep}{7} propagated by DTD and $\lrpp$ and averaged over MNIST \textit{Class 1} and FashionMNIST\textit{Class Trouser} test population.} 
\label{fig:exp1_dist_plot}
\end{figure}

\subsection{Summary}
Results from this preliminary experiment strongly support our hypothesis that  structure of RNN could have impact on the quality of  explanation.  In particular,  as presented in \addfigure{\ref{fig:class_1_comparison}} and \addfigure{\ref{fig:exp1_dist_plot}}, quality of DTD and $\lrpp$ explanation are significantly influenced by the architecture. In contrast, we do not see such notable effect from SA and GB method.  In the following, we are going to discuss similar experiments that are designed in a more constructive way. This construction enables us to methodologically evaluate the impact.


\section{Experiment 2 : Majority Sample Sequence Classification} \label{sec:exp2}
   
 \begin{figure}[!htb]
\centering
\includegraphics[width=0.8\textwidth]{sketch/artificial_problem_3digits}
\caption{Majority Sample Sequence Classification(MAJ) problem.} 
\label{fig:artificial_problem_3digits}
\end{figure}

\subsection{Problem Formulation} \label{sec:exp2_prob_formulate}
When neural networks are trained, one can apply explantation techniques to the models to get explanations of the outputs.  The explanation  of sample $\x$ indicates important features in $x$ that the trained network rely on to perform the objective task,  such as classification.  Therefore, one needs to know the ground truth where these latent features are in $\x$ in order to methodologically evaluate the explainability of the model.  However, this knowledge is not trivially available because it is an incident from the training process operating in high-dimensional space that we seek to understand in the first place.

To alleviate this challenge, we propose another artificial classification problem where RNN are trained to classify  the majority group in a sequence $\x$, $(\x_t )_{t=1}^{T}$. Consider MNIST. $\x$ is constructed as follows: for each original sample $\widetilde{\x} \in \mathbb{R}^{28,28}$, we randomly selected 2 additional samples : one from the same class of $\widetilde{\x}$ and the other one from a different class. Then, these 3 samples are concatenated in random order yielding a sample $\x \in \mathbb{R}^{28,84}$.  \addfigure{\ref{fig:artificial_problem_3digits}} illustrates the construction and the objective classification. Given $\x = \{ 8, 7, 8\}$, the classification result is ``8".  We call this problem as MNIST-MAJ when $\x$ are constructed from MNIST samples and the same to FashionMNIST-MAJ.

Because we already know blocks of digit/item that belong to the majority group from the construction,  we can then use this information to  quantitively evaluate  explanation quality  of each RNN.


As discussed in the previous experiment that only some DTD and $\lrpp$ explanations from the Deep architecture on FashionMNIST were sound. This seems to suggest that the architecture might not have enough capability to extract proper representations from FashionMNIST samples, causing the incorrect propagation issue.

Hence, apart of Shallow and Deep architecture, we are going to  introduce another two architectures, namely DeepV2 and ConvDeep. The DeepV2 architecture has one more layer after the first fully-connected layer than the Deep cell. On the other hand, the ConvDeep architecture instead replaces the first layer with  a sequence of convolutional and pooling operation. \addfigure{\ref{fig:deep_conv_arch}} shows details of these new architectures.
% todo : input padding


%\subsection{Setting}
%Two variations of \rnncell{Deep} cell are also experimented, namely \rnncell{DeepV2} and \rnncell{ConvDeep}, shown on \addfigure{\ref{fig:deep_conv_arch}}. The former has one additional layer \circled{1"} with dropout regularization  between \circled{1'}. On the other hand, the latter replaces fully connected layers between \circled{1} and \circled{3} with 2 convolutional and max pooling layers, \Big[\circled{C1}, \circled{P1}\Big] and \Big[\circled{C2},\circled{P2}\Big].

\begin{figure}[!htb]
\centering

\subfloat[DeepV2\label{fig:deep_4l_network}]{%
       \includegraphics[width=0.48\textwidth]{sketch/deep_v2_arch}
     }
     \hfill
     \subfloat[ConvDeep\label{fig:convdeep_4l_network}]{%
       \includegraphics[width=0.48\textwidth]{sketch/convdeep_arch}
     }

\patcaption{DeepV2 and ConvDeep architecture}{with number of neurons at each layer depicted.}
\label{fig:deep_conv_arch}
\end{figure}

Lastly, despite the fact that  our implementation is readily to apply on different sequence lengths,  we conducted the experiment with only sequence length $T=12$, more precisely $(\x_t \in \mathbb{R}^{28,7})_{t=1}^{12}$. This is mainly due to computational resources and time constraint we had. Consequently, we are going to write only the name of architecture without explicitly stating the sequence length as we previously proposed.


\subsection{Evaluation Methodology}
\label{sec:evaluation_med}
From the problem construction, we know that relevance quantities should  be primarily assigned to blocks that belong to the majority group. This construction enables us to both directly visually examine quality of explanations as well as performing quantitative evaluations.  In particular, for qualitative inspections, we constructed training and testing data based on the original training and testing split that \cite{LeCunMNISThandwrittendigit2010} and \cite{XiaoFashionMNISTNovelImage2017} proposed and trained with setting described in Section \ref{sec:setup}. 

\subsubsection{Quantitative Evaluation}
A straightforward way to quantitatively evaluate results is to calculate the percentage of relevance correctly distributed the blocks belonging to the majority group digit/item. However, this measurement has a shortcoming where architectures can achieve high score if they distribute relevance to only one of the correct blocks. Hence, we then propose to use \textit{cosine similarity} instead. The cosine similarity is computed from  a  binary  vector $\patvector{m} \in \mathbb{R}^3$  whose values represent correctness of the blocks and a vector $\patvector{\upsilon} \in \mathbb{R}^3$ containing percentage of  relevance distributed to the blocks. 

\begin{align}
\cos (\patvector{m}, \patvector{\upsilon}) = \frac{ \patvector{m} \cdot \patvector{\upsilon}}{ || \patvector{m}  ||_2 ||\patvector{\upsilon}   ||_2}	
\end{align}

As illustrated in \addfigure{\ref{fig:quantitative_evaluation}}, the percentage of correctly distributed relevance can be significantly high although the relevance heatmap does not show any highlight at the left most block of ``0". Therefore, using cosine similarity is more reasonable. In fact, the propagation needs to be equally balanced between the two blocks in order to achieve the highest score, ``1". For $\lrpp$ heatmaps, we ignore negative relevance and set it to zero before computing cosine similarity.

\begin{figure}[!htb]
\centering
\includegraphics[width=\textwidth]{sketch/quantitative_evaluation}
\patcaption{Comparison between percentage of correctly distributed relevance and cosine similarity.}{} 
\label{fig:quantitative_evaluation}
\end{figure}

To reduce variations possibly introduced by, for example variable initialization, we conducted quantitative evaluations through $k$-fold cross-validation process. We combined  training and testing set together and split the data into $k$ folds. Each fold is used as the testing set once. For each cross-validation iteration, we average the cosine similarity across testing samples. The final result is then averaged overall the folds' statistics.  To keep the same proportion of training and testing data, we chose $k=7$. 




% todo Statistical Evaluation
%\subsubsection{Statistical Evaluation}
%hypo : check whether we need it: 
%It is also possible that some architectures might perform similarly and the difference is not visually observed. For such scenarios, we will use one-way ANOVA on statistics of cosine similarity and pairwise Tukey Honest Significant Difference (HSD) as a post-hoc test to verify whether there are statistically significant results. We use significance level at $0.05$. Dataset is considered as a confounding variable. This procedure is conducted separately for each explanation method.
%}



 %Table x show accuracy sf

\subsection{Result}

\renewcommand{\arraystretch}{1.5}
\begin{table}[!hbt]
\begin{center}
\begin{tabular}{lc|c|c|}
\cline{3-4}
& &
\multicolumn{2}{c|}{\parbox{3.5cm}{ \vskip 1mm \centering \textbf{Accuracy} \vskip 1mm}} \\ \hline
\multicolumn{1}{|l|}{\textbf{Cell architecture}} & \textbf{No. variables} & \textbf{MNIST-MAJ} & \textbf{FashionMNIST-MAJ} \\ \hline
\multicolumn{1}{|l|}{Shallow}    & 184,330          & 98.12\% & 90.00\% \\ 
\multicolumn{1}{|l|}{Deep}       & 153,578           & 98.16\% & 89.81\% \\ 
 \multicolumn{1}{|l|}{DeepV2}     & 161,386        & 98.26\% & 90.57\% \\
\multicolumn{1}{|l|}{ConvDeep}   & 151,802       & 99.22\% & 92.87\%  \\ \hline 
\end{tabular}

\end{center}
\caption{Number of trainable variables and model accuracy from architectures trained on MNIST-MAJ and FashionMNIST-MAJ with sequence length $T=12$.}
\label{tab:maj_rnn_model_acc}
\end{table}
\renewcommand{\arraystretch}{1}

 \begin{figure}[!htb]
\centering
\includegraphics[width=\textwidth]{sketch/heatmap_msc_for_thesis}
\patcaption{Relevance heatmaps from different explanation techniques applied to Shallow, Deep, DeepV2 and ConvDeep architecture trained on MNIST-MAJ and FashionMNIST-MAJ with sequence length $T=12$.}{\heatmapscaleexplain } 
\label{fig:heatmap_msc_mix_for_thesis}
\end{figure}

Table \ref{tab:maj_rnn_model_acc} shows number of trainable variables and accuracy of the trained models. These trained models have equivalent number of variables and accuracy, except that ConvDeep has slightly higher accuracy. \addfigure{\ref{fig:heatmap_msc_mix_for_thesis}} shows that deeper architectures have better relevance propagation, in other words, they are more explainable in an aspect of prediction. In particular, we can see that portion of relevant scores distributed to irrelevant region are gradually reduced from Shallow to ConvDeep architecture. This effect happens across all explanation methods. This result further agrees with the evidence discussed in Section \ref{sec:exp1}.  

Although explanation heatmaps from \rnncell{Shallow}, \rnncell{Deep}, and \rnncell{DeepV2} generally look noisy, increasing the depth of architecture seems to reduce the noise in the heatmaps as well.   On the other hand, \rnncell{ConvDeep} does not  only properly assign relevance quantities to the right time steps, it also produces sound heatmaps containing features of $\x$ that are not easily observed from explanation from the other architectures.  GB and $\lrpp$ heatmaps of Digit 1 and Sandal sample are such examples.

\clearpage

 \begin{figure}[!hbt]
\centering
\includegraphics[width=\textwidth]{sketch/rel_dist_maj_3_samples_thesis}

\patcaption{Average cosine similarity from different explanation techniques and Shallow, Deep, DeepV2 and ConvDeep architecture.}{The values are averaged from cross-validation results and the vertical lines depicted 95\% confidence interval. The baseline is the Shallow architecture depicted by the blue line. Accuracy of the models can be found at Appendix  \ref{annex:model_acc}}


\label{fig:rel_dist_maj_3_samples_thesis}
\end{figure}

\addfigure{\ref{fig:rel_dist_maj_3_samples_thesis}} presents quantitive  evaluations of the impact from the depth of architecture to the quality of explanation. As a reminder, the measurement is cosine similarity between a mark vector $\boldsymbol{m} \in \mathbb{R}^3$ and an aggregated relevance to blocks of digit/item vector $\boldsymbol{\upsilon \in \mathbb{R}^3 }$ and averaged through $7$-fold cross-validation procedure as described in Section \ref{sec:evaluation_med}. Results from the figure indicate that the depth of architecture indeed improves quality of the explanations. In particular, the averaged cosine similarity of each explanation technique systematically increases when introducing  more layers. This effect can be seen clearly from the result of FashionMNIST-MAJ. Additionally, we can also observe that the difference of the metric between the baseline, \rnncell{Shallow}, and the other  architectures changes with different proposition across methods. In particular, we see the difference from DTD and $\lrpp$ are much larger than the other methods. This implies that some explanation methods are more sensitive to architecture configuration than the others.


% todo hypo : pair wise statistical testing

\subsection{Summary}
Results of this experiment quantitively confirm that architecture of RNN is indeed an important factor to the explainability of RNN models, especially in the aspect of propagating relevance to corresponding input steps.  The results also shows that this impact affects the quality of explanation in different level on different methods. More precisely, deep Taylor decomposition(DTD) and Layer-Wise Relevance Propagations(LRP) technique are more sensitive to the architecture of the explained model than sensitivity analysis(SA) and guided backprop(GB) method.

Nonetheless, we  also observed that there are some samples whose significant amount of relevance scores are distributed to irrelevant regions.  Digit ``9" sample on \addfigure{\ref{fig:heatmap_msc_mix_for_thesis}} is one of them. This issue is considerably obvious on ConvDeep.   Therefore, we are going to purpose several improvements to mitigate the issue.


		\section{Experiment 3 : Improve Relevance Distribution}

\subsection{Proposal 1 : Gating units}
\todo{Figure : R-LSTM}

\subsection{Proposal 2 :  Stationary Dropout}

 \begin{figure}[h]
\centering
\includegraphics[width=0.7\textwidth]{variational_dropout}
\caption{Variational Dropout\cite{GalTheoreticallyGroundedApplication2016}}
\label{fig:variational_dropout}
\end{figure}

\subsection{Proposal 3 : Convolutional layers with literal connections}

 \begin{figure}[h]
\centering
\includegraphics[width=0.7\textwidth]{sketch/conv_literalconn}
\caption{ConvDeep with Literal connections} 
\label{fig:conv_literalconn}
\end{figure}

\subsection{Result}

\subsection{Summary}


	
	\chapter{Conclusion}
\label{cha:chapter5}
%\section{Summary}
We have provided extensive experiments towards explaining RNN predictions. Our experiments are artificially designed such that qualitative and quantitative evaluations can be done accordingly.  The results demonstrate that the architecture of RNNs has a considerable impact on the quality of explanation. More precisely, we found that deeper and LSTM-type architectures have a great level of explainability.

Moreover, the level of influence from the RNN structure to the quality of explanation is different for each explanation technique. Based on our quantitative evaluations, the deep Taylor decomposition (DTD) and Layer-wise Relevance Propagation (LRP) techniques are more influenced by the RNN architecture than the sensitivity analysis (SA) and guided backprop (GB) methods.  Training configuration is also another influential factor that can affect the quality of explanation. In particular, for a certain architecture, training with stationary dropout shows a slight improvement in visual quality although our quantitative measurements do not capture the impact.

More importantly, it is worth mentioning that we consider ConvR-LSTM-SD as the most explainable architecture in this thesis. In particular, we achieve the decent explanation heatmaps when explaining it via $\lrpp$ without negative relevance considered. As a reminder, this heatmaps are shown in \addfigure{\ref{fig:heatmap_msc_convrlstm_pos_rel}}.

Lastly, we would like to argue further that the quality of RNN explanation should be considered in two aspects, namely fine-grained and coarse-grained.  The fine-grained aspect describes whether the explanation of each input from a sequence is sound. In case of image related applications, it is already shown in the literature that this aspect can be improved by employing convolutional and pooling layers. Our ConvDeep experiments confirm this in the RNN setting. On the other hand, the coarse-grained aspect tells us whether RNNs can adequately propagate relevance quantities to the right time steps in a sequence. Our experiments strongly suggest that using LSTM-type architecture is the key to improve the quality of explanation in this aspect. Therefore, RNNs need to satisfy these two aspects to establish great explainability.



\section{Challenges}
We have encountered several challenges while working on the thesis. Firstly, it is quite challenging to evaluate the quality of explanations when we do not have the ground truth information available. To mitigate this problem, we constructed artificial sequence classification problems such that we know parts of the input that are relevant to the objective.  Secondly, we have experienced that the initialization scheme of weights might also affect the quality of explanations although it does not affect the objective performance. In fact, some of the introduced architectures had worse explanations when weights were not initialized with the $1/\sqrt{N_{in}}$ scheme.

Lastly, because we only relied on basic frameworks, such as TensorFlow, and implemented most of the code ourselves, we found that implementing neural network systems is more challenging than traditional software development in a sense that we do not have a good way to verify the correctness of the code. Given that reason, we discovered that the conservation property is practically useful because it allows us to write unit tests that automatically check the implementation. This does not only enable us to validate new developments quickly, but it also makes sure that there will not be any systematic mistake in the implementation of LRP and DTD explanations for new architectures.

%hyper parameters... 

\section{Future work}
Despite results from our extensive experiments, we still consider our experimental setting limited, for example, we experimented using only a sequence length in some experiments.  Hence, one of future tasks would be to generalize and apply our work to a broader context. In particular,  applying the experiments on more diverse datasets and sequence lengths could be the first straightforward extension. Because of the popularity of RNNs in the NLP community, problems in this direction, such as text classification or sentiment analysis, are worth experimenting. 

As discussed earlier, quantifying the quality of RNN explanations is challenging in several aspects. We believe establishing a better quantitative evaluation methodology is another relevant future work.


% ---------------------------------------------------------------
\backmatter % no page numbering from here
    \input{./misc/acronyms}
		
		% if you want to provide a glossary with explanations of important terms put it in here

%    \bibliographystyle{geralpha}
    \bibliographystyle{apalike}
    \bibliography{./bib/references}
    
    \addchap{Appendix}

\begin{appendix}

\todo{appendix : all architectures that aren't desribed with no. neurons}

\todo{appendix : list of model accuracy and variance for hypothesis testing}

%\begin{figure}
%
%\caption{Hypothesis testing result for Experiment 1}
%\label{annex:hypo_exp1}
%\end{figure}

\lstset{captionpos=t,showstringspaces=false, basicstyle={\fontfamily{pcr}\selectfont\footnotesize}}

\lstinputlisting[label={annex:hypo_exp1}, caption={Hypothesis testing results of Section \ref{sec:exp2}}]{hypothesis-testing-results/exp1.txt}

\end{appendix}

\endinput


\end{document}
x